\documentclass[a4paper, 11pt]{article}

% -- Coments
\usepackage{verbatim}

% ----- Fonts -----
% -- Fuente --
% \usepackage{fontspec}
% \setmonofont{JetBrainsMono Nerd Font}  

% -- Color --
\usepackage{xcolor}
%\definecolor{azul}{RGB}{00,33,99}
\definecolor{azul}{RGB}{35,72,180}

% -- Page Margin --
\usepackage[margin=1in]{geometry}

% -- Espaciados --
\newcommand{\Pspace}{0.5cm}
\newcommand{\Aspace}{0.2cm}

% -- Columnas --
\usepackage{multicol}

% -- Imagenes --
\usepackage{graphicx}
\usepackage{float}

% -- Matemáticas --
\usepackage{amsmath, amssymb}

% -- Gráficas --
\usepackage{pgfplots}
\pgfplotsset{compat=1.18}

% -- Código --
\usepackage{listings}
\lstset{
    language=C++,                   % Lenguaje del código
    basicstyle=\ttfamily\small,     % Fuente del código
    keywordstyle=\color{blue},      % Color de palabras clave
    commentstyle=\color{gray},      % Color de comentarios
    stringstyle=\color{red},        % Color de cadenas
    numbers=left,                   % Números de línea a la izquierda
    numberstyle=\tiny\color{gray},
    breaklines=true,                % Permitir saltos de línea
    frame=single                    % Marco alrededor del código
}

\begin{comment}
-- Formato de pregunta simple
    % - Problema n
    \vspace{\Pspace}
    \item Problema \par
        % Respuesta:
        \vspace{\Aspace} \par
        { \color{azul}  }


 -- Formato de pregunta multimple
        % - Problema 
        \vspace{\Pspace}
        \item  \par
             % Respuestas:
            \vspace{\Aspace} \par
            a) 
            \\ { \color{azul} }

            \vspace{\Aspace} \par
            b) 
            \\ { \color{azul} }


 -- Formato de imagen
\begin{figure}[!ht]
    \centering
    \includegraphics[width=0.5\textwidth]{}
\end{figure}


 -- Formato de tabla
\begin{tabular}{c|c}
    \textbf{A} & \textbf{B} \\
    \hline
    Texto & Texto \\
    Texto & Texto
\end{tabular} \\
\end{comment}


\title
{
    Teoría de la computación 2026-1 \\
    $(L^{*})^{*} = L^{*}$
}

\begin{document}

    \maketitle

    \begin{center}
        \begin{tabular}{r|l}
            \textbf{Expediente} & \textbf{Nombre} \\ \hline
            223209096 & Ahumada Herrera Jorge Alán \\ 
            219208106 & Bórquez Guerrero Angel Fernando \\
            222206586 & Juan Valenzuela Gael \\
            223201053 & Solis Zatarain Owen Adiel \\
        \end{tabular}
    \end{center}

    \rule{\linewidth}{0.3mm}

    Ejercicio: demostrar $(L^{*})^{*} = L^{*}$ \\

    Sea $w \in L^{*}L^{*}$
    \begin{itemize}
        \item $w = xy$ donde $x \in L^{*} \land y \in L^{*}$
        \item $\exists m \geq 0 : x \in L^{m}$ y $\exists k > 0 : y \in L^{k}$
        \item $x = x_{1},...,x_{m}$, $x_{i} \in L$, $i = 1, ..., m$
        \item $y = y_{1},...,y_{k}$, $y_{j} \in L$, $i = 1, ..., k$
        \item $\therefore w \in L^{m+k} \subseteq L^{*}$
    \end{itemize}

    Sea $w \in L^{*} \land \varepsilon \in L^{*}$
    \begin{itemize}
        \item $w = w \varepsilon$
        \item Dado que $w \in L^{*}$ y $\varepsilon \in L^{*}$
        \item Entonces $w * \varepsilon \in L^{*}L^{*} : L^{*} \subseteq L^{*}L^{*}$
        \item $(L^{*})^{n} = L^{*} \forall n \geq 1$
    \end{itemize}

    Sea $P(n)$ la proposición $(L^{*})^{n} = L^{*}$. Demostremos que $P(n)$ es verdadera $\forall n \in \mathbb{N}, n \geq 1$
    \begin{itemize}
        \item Caso base: $n = 1$
        \item $(L^{*})^{1}$, sabemos que $L^{1} = L$, por lo que $(L^{*})^{1} = L^{*}$
        \item Hipótesis: $(L^{*})^{k} = L^{*}$
        \item Paso inductivo: $P(k + 1)$
        \item $(L^{*})^{k + 1} = L^{*}$
        \item $(L^{*})^{k}L^{*} = L^{*}$
        \item $L^{*}L^{*} = L^{*}$
        \item $(L^{*})^{n} = L^{*} \forall n \geq 1$
    \end{itemize}
    \textbf{QED}
\end{document}
