\documentclass[a4paper, 11pt]{article}

% -- Coments
\usepackage{verbatim}

% ----- Fonts -----
% -- Fuente --
% \usepackage{fontspec}
% \setmonofont{JetBrainsMono Nerd Font}  

% -- Color --
\usepackage{xcolor}
%\definecolor{azul}{RGB}{00,33,99}
\definecolor{azul}{RGB}{35,72,180}

% -- Page Margin --
\usepackage[margin=1in]{geometry}

% -- Espaciados --
\newcommand{\Pspace}{0.5cm}
\newcommand{\Aspace}{0.2cm}

% -- Columnas --
\usepackage{multicol}

% -- Imagenes --
\usepackage{graphicx}
\usepackage{float}

% -- Matemáticas --
\usepackage{amsmath, amssymb}

% -- Gráficas --
\usepackage{pgfplots}
\pgfplotsset{compat=1.18}

% -- Código --
\usepackage{listings}
\lstset{
    language=C++,                   % Lenguaje del código
    basicstyle=\ttfamily\small,     % Fuente del código
    keywordstyle=\color{blue},      % Color de palabras clave
    commentstyle=\color{gray},      % Color de comentarios
    stringstyle=\color{red},        % Color de cadenas
    numbers=left,                   % Números de línea a la izquierda
    numberstyle=\tiny\color{gray},
    breaklines=true,                % Permitir saltos de línea
    frame=single                    % Marco alrededor del código
}

\title
{
    Teoría de la Computación 2026-1 \\
    Ejercicio 7
}

    \begin{document}

    \maketitle

    \begin{center}
        \begin{tabular}{r|l}
            \textbf{Expediente} & \textbf{Nombre} \\ \hline
            223209096 & Ahumada Herrera Jorge Alán \\
            219208106 & Bórquez Guerrero Angel Fernando \\
            222206586 & Juan Valenzuela Gael \\
            223201053 & Solis Zatarain Owen Adiel \\
        \end{tabular}
    \end{center}

    \rule{\linewidth}{0.3mm}

    \vspace{0.3cm}

    \vspace{\Pspace}
    Sean $L$, $M$ y $N$ lenguajes cualesquiera, todos con un mismo alfabeto. Pruebe o de un contraejemplo de cada una de las siguientes afirmaciones:\par
    \begin{enumerate}
            \vspace{\Aspace} \par
            \item$L(M \cap N) = LM \cap LN$
            \\ { \color{azul} 
                Supongamos $L = \{u, uv\}$, $M = \{v\}$ y $N = \{\varepsilon \}$
                \par Se define a la concatenación de lenguajes como $LM = \{ xy : x \in L \land y \in M \}$
                \par reemplazamos los valores en la igualdad
                \par $\{ u, uv \}(\{ v \} \cap \{ \varepsilon \}) = \{ uv, uvv \} \cap \{ u, uv \}$
                \par $\{ u, uv \} \varnothing = \{ uv \}$
                \par $\varnothing = \{ uv \}$
                \par $\therefore L(M \cap N) \neq LM \cap LN$
            }

            \vspace{\Aspace} \par
            \item $L \cap (MN) = (L \cap M)(L \cap N)$
            \\ { \color{azul}  
                Supongamos $L = \{ u, v, w \}$, $M = \{ v \}$ y $N = \{ w \}$
                \par Reemplazamos en la igualdad
                \par $\{ u, v, w \} \cap \{ vw \} = (\{ u, v, w \} \cap \{ v \})(\{ u, v, w \} \cap \{ w \})$
                \par $\varnothing = (\{ v \})(\{ w \})$
                \par $\varnothing = \{ vw \}$
                \par $\therefore L \cap (MN) \neq (L \cap M)(L \cap N)$
            }

            \vspace{\Aspace} \par
            \item $LM = ML$
            \\ { \color{azul} 
                Supongamos $L = \{ u \}$ y $M = \{ v \}$
                \par Se define a la concatenación de lenguajes como $LM = \{ xy : x \in L \land y \in M \}$
                \par Entonces $LM = \{ uv \}$ y $ML = \{ vu \}$
                \par $\therefore LM \neq ML$
            }

            \newpage
                \item $LL^{*} = L^{*}L$
            \\ { \color{azul} 
                Parte 1: $LL^* \subseteq L^*L$
                \par Sea $w \in LL^*$.
                \par Por definición, $w = x \cdot y$ con $x \in L$ e $y \in L^*$.
                \par Por definición de cerradura de Kleene, existe $n \ge 0$ tal que $y \in L^n$.

                \par \textbf{Caso $n=0$ ($y = \varepsilon$):}
                \par $w = x \cdot \epsilon = x$
                \par $ w = \epsilon \cdot x \implies w \in L^0 L \subseteq L^*L$

                \par \textbf{Caso $n > 0$:}
                \par Por definición, $y = x_1 \dots x_n$ con $x_i \in L$ para $i=1,\dots,n$.
                \par Sustituyendo en $w$:
                \par \( w = x \cdot (x_1 \dots x_n) \)

                \par Por asociatividad de la concatenación:
                \par $ w = (x \cdot x_1 \dots x_{n-1}) \cdot x_n $

                \par Sea $z = x \cdot x_1 \dots x_{n-1}$.
                \par Notamos que $z$ está formado por la concatenación de $x$ (1 elemento) y $n-1$ elementos ($x_1 \dots x_{n-1}$).
                \par Total de elementos: $1 + (n-1) = n$.
                \par $ \therefore z \in L^n \subseteq L^* $

                \par Como $x_n \in L$, tenemos:
                \par $ w = z \cdot x_n \implies w \in L^*L $

                \par $ \therefore LL^* \subseteq L^*L $

                \par Parte 2: $L^*L \subseteq LL^*$

                \par Sea $w \in L^*L$.
                \par Por definición, $w = y \cdot x$ con $y \in L^*$ y $x \in L$.
                \par Por definición, existe $n \ge 0$ tal que $y \in L^n$.

                \par \textbf{Caso $n=0$ ($y = \epsilon$):}
                \par \( w = \epsilon \cdot x = x \)
                \par $ w = x \cdot \epsilon \implies w \in L L^0 \subseteq LL^* $

                \par \textbf{Caso $n > 0$:}
                \par Por definición, $y = x_1 \dots x_n$ con $x_i \in L$.
                \par Sustituyendo en $w$:
                \par $ w = (x_1 \dots x_n) \cdot x $

                \par Por asociatividad, separamos el primer elemento:
                \par $ w = x_1 \cdot (x_2 \dots x_n \cdot x) $

                \par Sea $z = x_2 \dots x_n \cdot x$. 
                \par Notamos que $z$ está formado por $n-1$ elementos ($x_2 \dots x_n$) más el elemento $x$.
                \par Total de elementos: $(n-1) + 1 = n$.
                \par $ \therefore z \in L^n \subseteq L^* $

                \par Como $x_1 \in L$, tenemos:
                \par $ w = x_1 \cdot z \implies w \in LL^* $

                \par $ \therefore L^*L \subseteq LL^* $


                \par Conclusión
                \par $ LL^* \subseteq L^L \quad \land \quad L^*L \subseteq LL^* \implies LL^* = L^*L \quad \blacksquare $
            }

            \vspace{\Aspace} \par
            \item Para todo $n > 0$, $(LM)^{n} = L^{n}M^{n}$.
            \\ { \color{azul} 
                Sea $L = \{ u \}$, $M = \{ v \}$ y $n = 2$
                \par $(LM)^{2} = L^{2}M^{2}$
                \par $(\{ u \}\{ v \})(\{ u \}\{ v \}) = \{ u \}\{ u \}\{ v \}\{ v \}$
                \par $\{ uvuv \} = \{ uuvv \}$
                \par $\therefore n > 0$, $(LM)^{n} \neq L^{n}M^{n}$
            }

            \vspace{\Aspace} \par
            \item Para todo $n$, $m > 0$, $(L^{n})^{m} = (L^{m})^{n}$.
            \\ { \color{azul} 
                Sea $w = u_{1}u_{2}...u_m$ con $u_{i} \in L^{n}$
                \par $w \in (L^{n})^{m}$
                \par Cada cadena $u_{i} = x_{i1}x_{i2}...x_{in}$ con $x_{ij} \in L$
                \par Por lo tanto $w = (x_{11}...x_{1n})(x_{21}...x_{2n})...(x_{m1}...x_{mn})$
                \par Esto es una concatenación de $nm$ cadenas de $L$, en este orden fijo
                \par Agrupamos ahora en bloques de tamaño $m$
                \par $w = (x_{11}x_{21}...x_{m1})(x_{12}x_{22}...x_{m2})...(x_{1n}x_{2n}...x_{mn})$
                \par Cada bloque es una concatenación de $m$ cadenas de $L$
                \par $(x_{1j}x_{2j}...x_{mj}) \in L^{m}$
                \par Y hay $n$ bloques
                \par Por definición $w \in (L^{m})^{n}$
                \par Entonces $(L^{n})^{m} \subseteq (L^{m})^{n}$
                \par Por simetría podemos decir que el mismo razonamiento al revés nos da $(L^{m})^{n} \subseteq (L^{n})^{m}$
                \par $\therefore n$, $m > 0$, $(L^{n})^{m} = (L^{m})^{n}$
            }
    \end{enumerate}
\end{document}
