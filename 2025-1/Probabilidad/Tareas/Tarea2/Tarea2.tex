\documentclass[a4paper, 12pt]{article}

% -- Language --
\usepackage[spanish]{babel}
\usepackage[utf8]{inputenc}

% ----- Fonts -----
% -- Color --
\usepackage{xcolor}
\definecolor{azul}{RGB}{51,102,204}

% -- Page Margin --
\usepackage[margin=1in]{geometry}

\title{%
  Probabilidad 2025-1 \\
  Tarea 2
}

\begin{document}

\maketitle

\begin{center}
    \begin{tabular}{r|l}
        \textbf{Expediente} & \textbf{Nombre} \\ \hline
        219208106 & Bórquez Guerrero Angel Fernando \\
        223203899 & Tostado Cortes Dante Alejandro \\
    \end{tabular}
\end{center}

\rule{\linewidth}{0.3mm}

\begin{enumerate}
    \item Una cerradura de combinación se abrirá cuando se seleccione la opción correcta de tres números (del 1 al 40, contando el 40). ¿Cuántas combinaciones diferentes de la cerradura son posibles?
    \vspace{0.2cm}
    \\ { \color{azul} $R/: 40^{3} = 64,000$ }

    \vspace{0.5cm}
    \item Cuatro parejas (8 personas) han reservado asientos en una fila para un concierto. ¿En cuántas formas pueden tomar asiento si (a) no hay restricciones para hacerlo? (b) los dos de cada pareja desean sentarse juntos?
    \vspace{0.2cm}
    \\ (a) { \color{azul} $R/: 8! = 40,320$ }
    \\ (b) { \color{azul} $R/: 4! \times 2^{4} = 384$ }
    
    \vspace{0.5cm}
    \item De entre un grupo de 40 personas, ha de seleccionarse un jurado de 12 personas. ¿En cuántas formas puede ser seleccionado el jurado?
    \vspace{0.2cm}
    \\ { \color{azul} $R/: C(40, 12) = \frac{40!}{(40-12)!12!} = 5,586,853,480$ }

    \vspace{0.5cm}
    \item La complejidad de las relaciones interpersonales crece de modo considerable cuando el tamaño del grupo es mayor. Determine el número de relaciones de dos parejas en grupos de (a) 3, (b) 8, (c) 12 y (d) 20 personas.
    \vspace{0.2cm}
    \\ (a) { \color{azul} $C(3, 2) = 3$ }
    \\ (b) { \color{azul} $C(8, 2) = 28$ }
    \\ (c) { \color{azul} $C(12, 2) = 66$ }
    \\ (d) { \color{azul} $C(20, 2) = 190$ }

    \vspace{0.5cm}
    \item ¿Cuántos arreglos diferentes pueden formarse con una caja de 8 colores, si contiene 4 blancos, 3 azules y 1 rojo?
    \vspace{0.2cm}
    \\ { \color{azul} $R:/ C(8, 4) \times C(4, 3) \times C(1, 1) = 280$ }

\newpage
    \vspace{0.5cm}
    \item Se sacan cinco cartas de un mazo ordinario de 52 cartas de juego. ¿Cuál es la probabilidad de que la mano sacada sea del mismo palo? ¿Cuál es la probabilidad de que la mano sacada sea una escalera? ¿Cuál es la probabilidad de que la mano sacada sea una tercia?


    \vspace{0.5cm}
    \item Se sacan cinco cartas de un mazo ordinario de 52 cartas de juego. ¿Cuál es la probabilidad de que la mano sacada sean dos pares? ¿Cuál es la probabilidad de que la mano sacada sea un par? ¿Cuál es la probabilidad de que la mano sacada no tenga ningún juego posible (carta alta)?
\end{enumerate}
\end{document}






% ana
