\documentclass[a4paper, 12pt]{article}

% -- Language --
\usepackage[spanish]{babel}
\usepackage[utf8]{inputenc}

% ----- Fonts -----
% -- Color --
\usepackage{xcolor}
%\definecolor{azul}{RGB}{00,33,99}
\definecolor{azul}{RGB}{35,72,180}

% -- Page Margin --
\usepackage[margin=1in]{geometry}

% -- Espaciados --
\newcommand{\Pspace}{0.5cm}
\newcommand{\Aspace}{0.2cm}

% -- Imagenes --
\usepackage{graphicx}

\title
{
  Probabilidad 2025-1 \\
  Tareas Parcial 2
}

\begin{document}

\maketitle

\begin{center}
    \begin{tabular}{r|l}
        \textbf{Expediente} & \textbf{Nombre} \\ \hline
        219208106 & Bórquez Guerrero Angel Fernando \\
        223203899 & Tostado Cortes Dante Alejandro \\
    \end{tabular}
\end{center}

\rule{\linewidth}{0.3mm}



% ---------- Tarea 5 ----------
\vspace{0.3cm}

\begin{center}
    { \LARGE Tarea 5}
\end{center}

\begin{enumerate}
    % - Problema 1
    \item Determinar si se trata de una variable aleatoria discreta o continua.
    % Respuestas:
    \vspace{\Aspace} \par
    a) $X$ = "Número de televisiones que prenden":
    \\ { \color{azul}  }

    \vspace{\Aspace} \par
    b) $Y$ = "Velocidad de un automóvil":
    \\ { \color{azul}  }
    
    \vspace{\Aspace} \par
    c) $Z$ = "Altura de una persona":
    \\ { \color{azul}  }

    \vspace{\Aspace} \par
    d) $W$ = "Número de llamadas telefónicas en alguna hora":
    \\ { \color{azul}  }

    \vspace{\Aspace} \par
    e) $X$ = "Mililitros en una lata de cerveza":
    \\ { \color{azul}  }


    % - Problema 2
    \vspace{\Pspace}
    \item Se lanaza una moneda 3 veces. Sea $X$ la variable aleatoria que cuenta el número de águilas que suceden en el experimento. Encontrar la función de probabilidad de la variable aleatoria y realizar el histograma.
    % Respuesta:
    \vspace{\Aspace} \par
    { \color{azul}  }
\end{enumerate}



% ---------- Tarea 6 ----------
\newpage
\begin{center}
    { \LARGE Tarea 2}
\end{center}

\begin{enumerate}
    %  - Problema 1
    \item Un dado equilibrado se lanza dos veces consecutivas. Sea $X$ la diferencia entre el resultado del primer y el segundo lanzamiento.
    % Respuestas:
    \vspace{\Aspace} \par
    • Encuentre la función de probabilidad de $X$.
    \\ { \color{azul}  }

    \vspace{\Aspace} \par
    • Encuentre la función de distribución de $X$.
    \\ { \color{azul}  }

    \vspace{\Aspace} \par
    • Determinar el valor esperado de $X$.
    \\ { \color{azul}  }

    \vspace{\Aspace} \par
    • Determinar la varianza de $X$.
    \\ { \color{azul}  }

    \vspace{\Aspace} \par
    • Determinar el histograma.
    \\ { \color{azul}  }

    \vspace{\Aspace} \par
    • Dibujar el valor esperado y el intervalo que resulta de estar dentro de una desviación estándar de la media.
    \\ { \color{azul}  }


    % - Problema 2
    \vspace{\Pspace}
    \item Se sabe que un jugador tiene una probabilidad del 95\% de ganar un juego importante. La variable aleatoria $X$, que indica si gana o pierde, ¿sigue una distribución de Bernoulli? Si pudiera jugar muchas veces el juego importante, ¿qué porcentaje de victorias esperaría observar? ¿Cuál es la varianza de esta variable aleatoria? Determine el intervalo de valores que están dentro de una desviación estándar de la media.
    % Respuestas:
    \vspace{\Aspace} \par
    { \color{azul}  }
\end{enumerate}

\end{document}
