\documentclass[a4paper, 12pt]{article}

% -- Language --
\usepackage[spanish]{babel}
\usepackage[utf8]{inputenc}

% ----- Fonts -----
% -- Fuente --
\usepackage{fontspec}
\setmonofont{JetBrainsMono Nerd Font}  

% -- Color --
\usepackage{xcolor}
%\definecolor{azul}{RGB}{00,33,99}
\definecolor{azul}{RGB}{35,72,180}

% -- Page Margin --
\usepackage[margin=1in]{geometry}

% -- Espaciados --
\newcommand{\Pspace}{0.5cm}
\newcommand{\Aspace}{0.2cm}

% -- Imagenes --
\usepackage{graphicx}
\usepackage{float}

% -- Matemáticas --
\usepackage{amsmath, amssymb}

% -- Gráficas --
\usepackage{pgfplots}
\pgfplotsset{compat=1.18}

% -- Código --
\usepackage{listings}
\lstset{
    language=C++,                   % Lenguaje del código
    basicstyle=\ttfamily\small,     % Fuente del código
    keywordstyle=\color{blue},      % Color de palabras clave
    commentstyle=\color{gray},      % Color de comentarios
    stringstyle=\color{red},        % Color de cadenas
    numbers=left,                   % Números de línea a la izquierda
    numberstyle=\tiny\color{gray},
    breaklines=true,                % Permitir saltos de línea
    frame=single                    % Marco alrededor del código
}


\title
{
    Probabilidad 2025-1 \\
    Tareas Parcial 4
}

    \begin{document}

    \maketitle

    \begin{center}
        \begin{tabular}{r|l}
            \textbf{Expediente} & \textbf{Nombre} \\ \hline
            219208106 & Bórquez Guerrero Angel Fernando \\
            223203899 & Tostado Cortes Dante Alejanro
        \end{tabular}
    \end{center}

    \rule{\linewidth}{0.3mm}



    % ---------- Tarea 13 ----------
    \vspace{0.3cm}

    \begin{center}
        { \LARGE Tarea 13}
    \end{center}

    \begin{enumerate}
        % - Problema 1
        \item Suponga que el tiempo promedio que le toma a una persona cualquiera terminar un videojuego es de 30 horas, con una desviación estándar de 5 horas. Suponiendo una distribución aproximada normal con estos parámetros, determine: \par
            % Respuestas:
            \vspace{\Aspace} \par
            a) La probabilidad de terminar el videojuego en menos de 20 horas.
            \\ { \color{azul} 
                $ P[X < 20] $ \par
                $ Z = \frac{X - \mu}{\sigma} = \frac{20 - 30}{5} = -\frac{10}{5} = -2 $ \par
                $ P[Z < -2] = 0{.}02275 = 2{.}275\% $
            }

            \vspace{\Aspace} \par
            b) La probabilidad de terminar el videojuego en menos de 40 horas.
            \\ { \color{azul} 
                $ P[X < 40] $ \par
                $ Z = \frac{X - \mu}{\sigma} = \frac{40 - 30}{5} = \frac{10}{5} = 2 $ \par
                $ P[Z < 2] = 0{.}97725 = 97{.}725\% $               
            }

            \vspace{\Aspace} \par
            c) La probabilidad de terminar el videojuego entre 20 y 40 horas.
            \\ { \color{azul} 
                $ P[20 < \leq X \leq 40] = P[X \leq 40] - P[X \leq 20] $ \par
                $ P[-2 \leq X \leq 2] = P[Z \leq 2] - P[Z \leq -2] $ \par
                $ 0{.}97725 - 0{.}02275 = 0{.}9545 = 95{.}45\% $
            }

            \vspace{\Aspace} \par
            d) El tiempo que tarda el 0.01\% de las personas que tardan menos en terminar el videojuego.
            \\ { \color{azul} 
                $ P[Z \leq z] = 0.00001 $ \par
                $ Z \approx -3{.}71902 $ \par
                $ X = Z\sigma + \mu = (-3{.}71902)(5) + 30 = -18{.}595 + 30 = 11{.}405 $
            }
    \end{enumerate}




    \newpage
    % ---------- Tarea 14 ----------
    \vspace{0.3cm}

    \begin{center}
        { \LARGE Tarea 14}
    \end{center}

    \begin{enumerate}
        % - Problema 1
        \item Describa con sus propias palabras. \par          
            % Respuestas:
            \vspace{\Aspace} \par
            a) La ley de los grandes números. 
            \par { \color{azul} Esta ley dice que al tomar una muestra aleatoria de un universo esta tendrá a aestar cerca de la media de ese universo. Esto nos dice que relativamente podemos calcular la media de un universo con pocos datos. }

            \vspace{\Aspace} \par
            b) El teorema del límite central.
            \par { \color{azul} Este teorema nos dice que para una muestra cualquiera de una población con $n$ elementos en ella tiene una distribución normal estándar a medida que $n$ tiende al infinito }
        

        % - Problema 2
        \item Demuestre que las siguientes funciones son de densidad.
            % Respuestas:
            \par a)
            \[
                f(x, y) =
                \begin{cases}
                    e^{-(x + y)}    &   x , y > 0           \\
                    0               &   \text{Otro caso}
                \end{cases}
            \]
            \vspace{\Aspace}
            { \color{azul} 
                \\ • \( 
                    \int_{0}^{\infty} \int_{0}^{\infty} e^{-(x + y)}dydx
                    = \int_{0}^{\infty} \int_{0}^{\infty} e^{-x-y}dydx
                    = \int_{0}^{\infty} e^{-x} \int_{0}^{\infty} e^{-y}dydx
                \)

                • \(
                    \int_{0}^{\infty} e^{-y}dy
                    = -e^{-y} \Big|_{0}^{\infty}
                    = \lim_{y \rightarrow \infty} -\frac{1}{e^{y}} - (e^{0})
                    = 0 - (-1)
                    = 1
                \)

                • \(
                    \int_{0}^{\infty} e^{-x} \int_{0}^{\infty} e^{-y}dydx
                    = \int_{0}^{\infty} e^{-x}(1)dx
                    = \int_{0}^{\infty} e^{-x}dx
                \)

                • \(
                    \int_{0}^{\infty} e^{-x}dx
                    = -e^{-x} \Big|_{0}^{\infty}
                    = \lim_{x \rightarrow \infty} -\frac{1}{e^{x}} - (e^{0})
                    = 0 - (-1)
                    = 1
                \)

                Dado que el resultado de la integral de la función $f(x, y)$ es igual a 1 hemos demostrado que es una función de densidad.
            }

            \par b)
            \[
                f(x, y) =
                \begin{cases}
                    3xy(1 - x)      &   0 < x < 1, 0 < y < 2    \\
                    0               &   \text{Otro caso}
                \end{cases}
            \]
            \vspace{\Aspace}
            { \color{azul} 
                \\ • \(
                    \int_{0}^{1} \int_{0}^{2} 3xy(1 - x)dydx
                    = \int_{0}^{1} 3x(1 - x) \int_{0}^{2} ydydx
                \)

                • \(
                    \int_{0}^{2} ydydx
                    = \frac{y^{2}}{2} \Big|_{0}^{2} 
                    = \frac{2^{2}}{2}
                    = \frac{4}{2}
                    = 2
                \)

                • \(
                    \int_{0}^{1} 3x(1 - x) \int_{0}^{2}ydydx
                    = \int_{0}^{1} 3x(1 - x)(2)dx
                    = \int_{0}^{1} 6x-6x^{2}
                    = 6 \left[ \int_{0}^{1}xdx - \int_{0}^{1}x^{2}dx \right]
                \)

                • \(
                    \int_{0}^{1}xdx
                    = \frac{x^{2}}{2} \Big|_{0}^{1}
                    = \frac{1^{2}}{2} - \frac{0^{2}}{2} 
                    = \frac{1}{2}
                \)

                • \(
                    \int_{0}^{1}xdx
                    = \frac{x^{3}}{3} \Big|_{0}^{1}
                    = \frac{1^{3}}{3} - \frac{0^{3}}{3} 
                    = \frac{1}{3}
                \)

                • \(
                    6 \left[ \frac{1}{2} - \frac{1}{3} \right]
                    = 6 \left[ \frac{1}{6} \right]
                    = 1
                \)

                Dado que el resultado de la integral de la función $f(x, y)$ es igual a 1 hemos demostrado que es una función de densidad.

            }
    \end{enumerate}



    \newpage
    % ---------- Tarea 15 ----------
    \vspace{0.3cm}

    \begin{center}
        { \LARGE Tarea 15}
    \end{center}

    \begin{enumerate}
        % - Problema 1
        \item Calcular las funciones de densidades marginales $f_{X}(x)$ y $f_{Y}(y)$
        \[
            f(x, y) =
            \begin{cases}
                4xy     &   0 < x, y < 1    \\
                0       &   \text{Otro caso}
            \end{cases}
        \]
            % Respuesta:
            \vspace{\Aspace}
            { \color{azul}  
                \par • Densidad marginal de $f_{X}(x)$ \\
                \(
                    f_{X}(x)
                    = \int_{0}^{1} 4xydy
                    = 4x \int_{0}^{1} ydy
                    = 4x \left[ \frac{y^{2}}{2} \Big|_{0}^{1} \right]
                    = 4x \left[ \frac{1^{2}}{2} - \frac{0^{2}}{2} \right]
                    = 4x \left[ \frac{1}{2} \right]
                    = 2x
                \)
                \[
                    f_{X}(x) =
                    \begin{cases}
                        2x  &   0 < x < 1 \\
                        0   &   \text{Otro caso}
                    \end{cases}
                \]

                \par • Densidad marginal de $f_{Y}(y)$ \\
                \(
                    f_{Y}(y)
                    = \int_{0}^{1} 4xydy
                    = 4y \int_{0}^{1} xdx
                    = 4y \left[ \frac{x^{2}}{2} \Big|_{0}^{1} \right]
                    = 4y \left[ \frac{1^{2}}{2} - \frac{0^{2}}{2} \right]
                    = 4y \left[ \frac{1}{2} \right]
                    = 2y
                \)
                \[
                    f_{Y}(y) =
                    \begin{cases}
                        2y  &   0 < y < 1 \\
                        0   &   \text{Otro caso}
                    \end{cases}
                \]
            }
        

        % - Problema 2
        \item Encuentre las funciones de distribución marginales $F_{X}(x)$ y $F_{Y}(y)$
        \[
            F(x, y) =
            \begin{cases}
                (1 - e^{-x})(1 - e^{-y})    &   x, y > 0    \\
                0                           &   \text{Otro caso}
            \end{cases}
        \]
            % Respuesta:
            \vspace{\Aspace}
            { \color{azul} 
                \par • Distribución marginal de $f_{X}(x)$ \\
                \(
                    f_{X}(x)
                    = \lim_{y \rightarrow \infty} (1 - e^{-x})(1 - \frac{1}{e^{y}})
                    = (1 - e^{-x})(1 - \frac{1}{e^{\infty}})
                    = (1 - e^{-x})(1 - 0)
                    = (1 - e^{-x})(1)
                    = 1 - e^{-x}
                \)
                \[
                    f_{X}(x) =
                    \begin{cases}
                        1 - e^{-x}  &   x > 0       \\
                        0           &   x \leq 0
                    \end{cases}
                \]

                \par • Distribución marginal de $f_{Y}(y)$ \\
                \(
                    f_{Y}(y)
                    = \lim_{x \rightarrow \infty} (1 - \frac{1}{e^{x}})(1 - e^{-y})
                    = (1 - \frac{1}{e^{\infty}})(1 - e^{-y})
                    = (1 - 0)(1 - e^{-y})
                    = (1)(1 - e^{-y})
                    = 1 - e^{-y}                   
                \)
                \[
                    f_{Y}(y) =
                    \begin{cases}
                        1 - e^{-y}  &   y > 0       \\
                        0           &   y \leq 0
                    \end{cases}
                \]
            }
    \end{enumerate}



    \newpage
    % ---------- Tarea 16 ----------
    \vspace{0.3cm}

    \begin{center}
        { \LARGE Tarea 16}
    \end{center}

    \begin{enumerate}
        % - Problema 1
        \item Un equipo de fútbol puede tener uno de los siguientes estados tras cada partido. 
        \\G: gana el partido, 
        \\P: pierde el partido. 
        \\La probabilidad de ganar o perder depende del resultado del partido anterior, de la siguiente manera: \par
        • Si gana un partido, tiene una probabilidad de 0.7 de ganar el siguiente y 0.3 de perder. \par
        • Si pierde un partido, tiene una probabilidad de 0.4 de ganar el siguiente y 0.6 de perder.
            % Respuestas:
            \vspace{\Aspace} \par
            a) Escribe la matriz de transición de esta cadena de Markov.
            \\ { \color{azul} 
                \[
                    \begin{array}{ccc}
                        \begin{bmatrix}
                            P(G \rightarrow G)  &   P(G \rightarrow P)  \\
                            P(P \rightarrow G)  &   P(P \rightarrow P)  
                        \end{bmatrix}
                    
                        =
                    
                        \begin{bmatrix}
                            0{.}7   &   0{.}3   \\
                            0{.}4   &   0{.}6
                        \end{bmatrix}
                    \end{array}
                \]
            }

            \vspace{\Aspace} \par
            b) Si el equipo ganó el primer partido, ¿cuál es la probabilidad de que gane el tercer partido?
            \\ { \color{azul} 
                • Vector inicial \\
                \(
                    \begin{array}{ccc}
                        v_{1}   
                        =
                        \begin{bmatrix}
                            1   &   0
                        \end{bmatrix}
                    \end{array}
                \)

                • Probabilidad del segundo partido \\
                \(
                    \begin{array}{cccccc}
                        v_{2}
                        
                        =
                        
                        \begin{bmatrix}
                            1   &   0
                        \end{bmatrix}
                        
                        \begin{bmatrix}
                            0{.}7   &   0{.}3   \\
                            0{.}4   &   0{.}6
                        \end{bmatrix}
                        
                        =

                        \begin{bmatrix}
                            0{.}7   &   0{.}3
                        \end{bmatrix}
                    \end{array}
                \)

                • Probabilidad del tercer partido \\
                \(
                    \begin{array}{cccccc}
                        v_{3}

                        =

                        \begin{bmatrix}
                            0{.}7   &   0{.}3
                        \end{bmatrix}

                        \begin{bmatrix}
                            0{.}7   &   0{.}3   \\
                            0{.}4   &   0{.}6
                        \end{bmatrix}

                        =

                        \begin{bmatrix}
                            0{.}61  &   0{.}39
                        \end{bmatrix}
                    \end{array}
                \)
            }

            \vspace{\Aspace} \par
            c) ¿Cuál es la distribución estacionaria?
            \\ { \color{azul} 
                \(
                    \\ $ \pi P = \pi $ \\
                    \begin{array}{cccc}
                        \begin{bmatrix}
                            \pi G   &   \pi P
                        \end{bmatrix}
                    
                        \begin{bmatrix}
                            0{.}7   &   0{.}3   \\
                            0{.}4   &   0{.}6                       
                        \end{bmatrix}

                        =

                        \begin{bmatrix}
                            \pi G   &   \pi P
                        \end{bmatrix}                   
                    \end{array}
                \)
            }

            \vspace{\Aspace} \par
            d) ¿Cuál es la distribución de probabilidad después de muchos partidos?
            \\ { \color{azul}  }

        
        \newpage
        % - Problema 2
    \item Un jugador participa en un juego en el que, en cada turno, gana o pierde 25 pesos con probabilidad $p = 0{.}5$. El capital del jugador puede tomar los valores $\{0, 25, 50, 75\}$, donde si alcanza los 75 pesos, el jugador gana definitivamente y deja de jugar. Si su capital llega a 0, se arruina y también deja de jugar. Mientras tenga 25 o 50 pesos, el juego continúa.
            % Respuestas:
            \vspace{\Aspace} \par
            a) Modele esta situación como una cadena de Markov y escriba su matriz de transición, considerando los estados en el orden 0, 25, 50, 75.
            \\ { \color{azul}  }

            \vspace{\Aspace} \par
            b) ¿Cuál es la probabilidad de que el jugador se haya arruinado después de tres juegos si comienza con 25 pesos?
            \\ { \color{azul}  }

            \vspace{\Aspace} \par
            c) ¿Existe una distribución estacionaria?
            \\ { \color{azul}  }

            \vspace{\Aspace} \par
            d) ¿Existe una distribución límite para esta cadena de Markov?
            \\ { \color{azul}  }
    \end{enumerate}
\end{document}
