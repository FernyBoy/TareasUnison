\documentclass[a4paper, 12pt]{article}

% -- Language --
\usepackage[spanish]{babel}
\usepackage[utf8]{inputenc}

% ----- Fonts -----
% -- Color --
\usepackage{xcolor}
\definecolor{azul}{RGB}{51,102,204}

% -- Page Margin --
\usepackage[margin=1in]{geometry}

\title{%
  Probabilidad 2025-1 \\
  Tarea 1
}

\begin{document}

\maketitle

\begin{center}
    \begin{tabular}{r|l}
        \textbf{Expediente} & \textbf{Nombre} \\ \hline
        219208106 & Bórquez Guerrero Angel Fernando \\
        223203899 & Tostado Cortes Dante Alejandro \\
    \end{tabular}
\end{center}

\rule{\linewidth}{0.3mm}

\begin{enumerate}
    \item Si $n(A) = 15$, $n(B) = 20$, y $n(A \cap B) = 10$. Encontrar $n(A \cup B).$
    \\ { \color{azul} $n(A \cup B) = n(A) + n(B) - n(A \cap B)$ }
    \\ { \color{azul} $n(A \cup B) = 15 + 20 - 10$ }
    \\ { \color{azul} $n(A \cup B) = 25$ }

    \vspace{0.5cm}
    \item Una persona tiene 4 camisetas, 5 pantalones, 7 pares de zapatos y 2 sombreros. ¿Cuántos conjuntos diferentes podría vestir?
    \\ { \color{azul} $R/: 4 \times 5 \times 7 \times 2 = 280$ }

    \vspace{0.5cm}
    \item ¿Cuántos números de tres dígitos se forman utilizando los dígitos 0, 1, 2, 3, 4, 5, 6, 7, 8 y 9? Se permiten dígitos repetidos.
    \\ { \color{azul}$R/: 10 \times 10 \times 10 = 1000$ } 
\end{enumerate}
\end{document}
