\documentclass[a4paper, 12pt]{article}

% -- Language --
\usepackage[spanish]{babel}
\usepackage[utf8]{inputenc}

% ----- Fonts -----
% -- Color --
\usepackage{xcolor}
\definecolor{azul}{RGB}{51,102,204}
% -- Family --
% \usepackage{fontspec}
% \setmainfont{JetBrainsMono Nerd Font}

% -- Page Margin --
\usepackage[
    top = 0.5in,
    bottom = 1in,
    left = 1in,
    right = 1in
]{geometry}

% -- Enumerates --
\usepackage{enumitem}

\title{%
  Teoría de la computación 2025-1 \\
  Lema de Arden
}

\begin{document}

\maketitle

\begin{center}
    \begin{tabular}{r|l}
        \textbf{Expediente} & \textbf{Nombre} \\ \hline
        223210350 & Amaya Soria Angel Alberto \\
        219208106 & Bórquez Guerrero Angel Fernando \\
        223217959 & Figueroa Torres Oiram Alberto \\
    \end{tabular}
\end{center}

\rule{\linewidth}{0.3mm}

Pruebe el lema de Arden. Considere la ecuación entre lenguajes $X = XM \cup N$ con $X$ desconocida. Se tiene lo siguiente:
\begin{enumerate}
    \item $X_{0}$ es solución de la ecuación $X = XM \cup N$.
    \begin{enumerate}[label = -]
                    \item Reemplazamos en la ecuacuión:
                    \\ $NM^{*} = NM^{*}M \cup N$

                    \item Sacamos factor común $N$:
                    \\ $NM^* = N(M^{*}M \cup \{ \varepsilon \})$

                    \item Utilizamos la propiedad $\Sigma^{*}\Sigma = \Sigma^{+}$:
                    \\ $NM^{*} = N(M^{+} \cup \{ \varepsilon \})$

                    \item Dado que $\Sigma^{+} \cup \{ \varepsilon \} = \Sigma^{*}$ reemplazamos:
                    \\ $NM^{*} = N(M^{*})$

                    \item Eliminamos paréntesis:
                    \\ $NM^{*} = NM^{*}$
                \end{enumerate}

    \vspace{0.5cm}
    \item Si $L$ es otra solución $X = XM \cup N$ entonces $X_{0} \subset L$. Esto es, $X_{0}$ es la solución más pequeña de $X = XM \cup N$.
    \begin{enumerate}[label = -]
        \item Suponemos que $L$ es solución, por lo tanto:
        \\ $L = LM \cup N$

        \item Decimos que $\varepsilon \in M$ ya que $L$ es solución.

        \item Existe una $w \in X_{0}$, y ya que $X_{0} \subset L$, entonces $w \in L$.

        \item Si $w \in X_0$, entonces $w \in NM^{n}$ para algún $n = 1, 2, 3, ...$.

        \item Definimos la función $P(n)$ como $w \in NM^{n}$.

        \item Evaluamos para 0 como caso base y para $n + 1$:
        \\ $P(0) = w \in NM^{0} = w \in N\{ \varepsilon \} = w \in N$
        \\ $P(n + 1) = w \in NM^{n + 1} = w \in NM^{n}M$

        \item Desarrollamos $w$ de tal forma que $w = xy | x \in NM^{n}$ y $y \in M$.

        \item Por suposición, $x \in L$.
        
        \item Tenemos que $w \in LM$, entonces $w \in LM \cup N$ y por lo tanto $w \in L$
    \end{enumerate}


    \newpage


    \vspace{0.5cm}
    \item Si $\varepsilon \in M$ entonces para cualuiqer lenguaje $S$, $X_{S}=(N \cup S)M^{*}$ es solución de $X = XM \cup N$. Sugerencia: note que si $\varepsilon \in M$ entonces $M^{*}M = M^{*}$,
    \begin{enumerate}[label = -]
        \item Por demostrar que $(N \cup S)M^{*} = (N \cup S)M^{*}M \cup N$

        \item Suponemos $\varepsilon \in M$

        \item Sea $w \in (N \cup S)M^{*}$, como $\varepsilon \in M$ entonces $w \in (N \cup S)M^{*}M = (N \cup S)M^{*}$, luego $w \in (N \cup S)M^{*}M \cup N$.

        \item Con esto $(N \cup S)M^{*} \subset (N \cup S)M^{*}M \cup N$

        \item Sea $w \in (N \cup S)M^{*}M \cup N$

        \item Si $w \in (N \cup S)M^{*}M$ se cumple que $w \in (N \cup S)M^{*}$ ya que $(N \cup S)M^{*}M = (N \cup S)M^{*}$

        \item Si $w \in N$ entonces $w \in N\{ \varepsilon \}$

        \item $w \in NM^{0}$ entonces $w \in NM^{*}$ y $w \in NM^{*} \cup SM^{*}$ pero con $NM^{*} \cup SM^{*} = (N \cup S)M^{*} = X_{S}$,

        \item Luego $w \in X_{S}$, con esto $(N \cup S)M^{*}M \cup N \subset (N \cup S)M^{*}$
    \end{enumerate}


    \vspace{0.5cm}
\item Si $\varepsilon \notin M$ entonces $X_{0}$ es la única solución de $X = XM \cup N$. Sugerencia: sea $L$ otra solución de $X = XM \cup N$, pruebe que para todo $n > 0$, $L - LM^{n + 1} \cup N \cup^{n}_{i = 0} M^{i}$; pruebe que si $w \in L$ con $|w| = n$ entonces $w$ no puede estar en $LM^{n + 1}$ y por tanto tiene que estar en $N \cup^{n}_{i = 0} M^{i} \subset X_{0}.$
    \begin{enumerate}[label = -]
        \item Supongamos $\varepsilon \notin M$ y que $L$ es otra solución, entonces $L$ tiene la forma:
        \\ $L = LM \cup N$, pero si reemplazamos $L$, $L = (LM \cup N)M \cup N = LM^{2} \cup NM \cup N = LM^{2} \cup N \cup^{n}_{i = 0}M^{i}$
        
        \item Proposición: $L$ tiene la forma $L = LM^{n + 1} \cup N \cup^{n}_{i = 0}M^{i}, \forall n > 0$.
        \\ $P(1): L = LM^{2} \cup N \cup NM = (LM \cup N)M \cup N = LM \cup N$

        \item Supongamos que $L = LM^{N + 1} \cup N \cup^{n}_{i = 0}M^{i}$
        
        \item Demostrar $L = LM^{n + 2} \cup N \cup^{n}_{i = 0}M^{i}$
        \\ $ = LM^{n + 2} \cup N \cup NM \cup ... \cup NM^{n + 1}$
        \\ $ = LM^{n + 2} \cup NM \cup ... \cup NM^{n + 1}$
        \\ $ = (LM^{n + 1} \cup N \cup ... \cup NM^{n})M \cup N$
        \\ $ = (LM^{n + 1} \cup N \cup^{n}_{i = 0}M^{i})M \cup N$
        \\ Y por suposición $LM \cup N$

        \item $P(n)$: si $w \in L$ y $|w| = n$ entonces $w \notin LM^{n + 1}$ y $w \in N  \cup^{n}_{i = 0}M^{i}$

        \item $P(0)$: $w = \varepsilon \in L \rightarrow |w| = 0$ ahora como $L = LM \cup N$, si $w \in LM$ y $\varepsilon \notin M$ entonces $\exists \alpha \in M \rightarrow |\alpha| \geq 1$, luego $|w| \geq 1$ lo cual contradice a que $|w| = 0$ entonces $w \in N = NM^{0} \subset X_{0}$

        \item $P(n > 0)$ sea $w \in L$ entonces $|w| = n > 0$, luego $w \in LM^{n + 1} \cup N \cup^{n}_{i = 0}M^{i}$, pero si $w \in LM^{n + 1}$ entonces $|w| \geq n + 1$ lo cual contradice $|w| = n$ entonces necesariamente $w \in N \cup^{n}_{i = 0}M^{i} \subset X+{0}$

        \item Demostramos que si $\varepsilon \notin M$ entonces $\forall w \in L \rightarrow w \in X_{0}$, es decir, $L \subset X_{0}$, ademas por la suposición que $L$ es solución distinta y por la demostración (2) tenemos que $X_{0} \subset L$. 
        \\ Concluyendo que $X_{0} = L$, es decir $X_{0}$ es única.
    \end{enumerate}


\end{enumerate}
\end{document}
