\documentclass{standalone}

% Paquetes necesarios
\usepackage{circuitikz}
\usetikzlibrary{circuits.logic.IEC}

\begin{document}
\begin{circuitikz}
\tikzstyle{every node}=[font=\LARGE]
\node [font=\LARGE] at (4,15.5) {M};
\node [font=\LARGE] at (4,14.25) {N};
\node [font=\LARGE] at (4,14) {};
\node [font=\LARGE] at (4,13) {Q};
\node at (9.75,14.25) [circ] {};
\node [font=\LARGE] at (10,14.25) {x};
\draw (5,15.5) to[short] (5.25,15.5);
\draw (5,15) to[short] (5.25,15);
\draw (5,15.5) node[ieeestd or port, anchor=in 1, scale=0.89](port){} (port.out) to[short] (7,15.25);
\draw (4.5,15.5) to[short] (5,15.5);
\draw (4.5,14.25) to[short] (5,14.25);
\draw (5,14.25) to[short] (5,15);
\draw (7.5,14.5) to[short] (7.75,14.5);
\draw (7.5,14) to[short] (7.75,14);
\draw (7.75,14.5) node[ieeestd and port, anchor=in 1, scale=0.89](port){} (port.out) to[short] (9.5,14.25);
\draw (7,15.25) to[short] (7,14.5);
\draw (7,14.5) to[short] (7.5,14.5);
\draw (8,14) to[short] (7,14);
\draw (7,14) to[short] (7,13);
\draw (7,13) to[short] (4.5,13);
\end{circuitikz}
\end{document}
