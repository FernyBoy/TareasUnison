\documentclass[a4paper, 11pt]{article}

% -- Language --

% -- Coments
\usepackage{verbatim}

% ----- Fonts -----
% -- Fuente --
% \usepackage{fontspec}
% \setmonofont{JetBrainsMono Nerd Font}  

% -- Color --
\usepackage{xcolor}
%\definecolor{azul}{RGB}{00,33,99}
\definecolor{azul}{RGB}{35,72,180}

% -- Page Margin --
\usepackage[margin=1in]{geometry}

% -- Espaciados --
\newcommand{\Pspace}{0.5cm}
\newcommand{\Aspace}{0.2cm}

% -- Columnas --
\usepackage{multicol}

% -- Imagenes --
\usepackage{graphicx}
\usepackage{float}

% -- Matemáticas --
\usepackage{amsmath, amssymb}
\usepackage{mathtools}

% -- Gráficas --
\usepackage{pgfplots}
\pgfplotsset{compat=1.18}

% -- Código --
\usepackage{listings}
\lstset{
    language=C++,                   % Lenguaje del código
    basicstyle=\ttfamily\small,     % Fuente del código
    keywordstyle=\color{blue},      % Color de palabras clave
    commentstyle=\color{gray},      % Color de comentarios
    stringstyle=\color{red},        % Color de cadenas
    numbers=left,                   % Números de línea a la izquierda
    numberstyle=\tiny\color{gray},
    breaklines=true,                % Permitir saltos de línea
    frame=single                    % Marco alrededor del código
}

\usepackage{enumerate}

\begin{comment}
-- Formato de pregunta simple
    % - Problema n
    \vspace{\Pspace}
    \item Problema \par
        % Respuesta:
        \vspace{\Aspace} \par
        { \color{azul}  }


 -- Formato de pregunta multimple
        % - Problema 
        \vspace{\Pspace}
        \item  \par
             % Respuestas:
            \vspace{\Aspace} \par
            a) 
            \\ { \color{azul} }

            \vspace{\Aspace} \par
            b) 
            \\ { \color{azul} }


 -- Formato de imagen
\begin{figure}[!ht]
    \centering
    \includegraphics[width=0.5\textwidth]{}
\end{figure}


 -- Formato de tabla
\begin{tabular}{c|c}
    \textbf{A} & \textbf{B} \\
    \hline
    Texto & Texto \\
    Texto & Texto
\end{tabular} \\
\end{comment}


\title
{
    Análisis Lógico 2025-2 \\
    Tarea pre-examen Unidad II
}

    \begin{document}

    \maketitle

    \begin{center}
        \begin{tabular}{r|l}
            \textbf{Expediente} & \textbf{Nombre} \\ \hline
            219208106 & Bórquez Guerrero Angel Fernando \\
        \end{tabular}
    \end{center}

    \rule{\linewidth}{0.3mm}
    
    
\begin{enumerate}[I.]
    \item Responde si las siguientes afirmaciones son verdaderas (V) o falsas (F), justificando en una línea si es falso (corrige la afirmación)
    \begin{enumerate}[1.]
        \item {\color{azul}\textbf{[V]}} En un lenguaje de primer orden no puede haber símbolos de función de aridad 0.


        \item {\color{azul}\textbf{[V]}} Toda constante es, desde el punto de vista sintáctico, un término.


        \item {\color{azul}\textbf{[F]}} El alcance de un cuantificador $\forall x$ o $\exists x$ es toda la fórmula en la que aparece.
        \\ { \color{azul} El alcance se limita a la subfórmula que sigue al cuantificador. }


        \item {\color{azul}\textbf{[F]}} El alfabeto de primer orden se compone de variables, constantes, funciones, predicados, conectivos lógicos, símbolos especiales.
        \\ { \color{azul} Faltan los cuantificadores. }


        \item {\color{azul}\textbf{[F]}} Los predicados se representan con una letra minúscula (Ej. $p(x) = $ ``$x$ es par").
        \\ { \color{azul} La letra debe de ser mayúscula. }


        \item {\color{azul}\textbf{[F]}} La aridad representa el número de predicado o función sobre el cual estamos hablando.
        \\ { \color{azul} La aridad es el número de elementos (objetos o variables) sobre los que un predicado o una función se aplica. }


        \item {\color{azul}\textbf{[V]}} $Ama(x,y) = $ ``$x$ ama a $y$" es un ejemplo de aridad 2.


        \item {\color{azul}\textbf{[F]}} Un término es cualquier expresión que ``nombra" un objeto del dominio. Nos dice si algo es verdadero o falso.
        \\ { \color{azul} Aunque es verdad que un término nombra un objeto del dominio, lo que nos dice si algo es verdadero o falso es un predicado. }


        \item {\color{azul}\textbf{[F]}} Un predicado es un ejemplo de término.
        \\ { \color{azul} Un predicado dice si algo es verdadero o falso, los términos representan objetos del dominio.  }


        \item {\color{azul}\textbf{[V]}} Una fórmula atómica (o átomo) es un predicado aplicado a un término.


        \item {\color{azul}\textbf{[F]}} LPO significa: Lógica de primer orden.
        \\ { \color{azul} LPO significa: Lenguaje de Primer Orden. }


        \item {\color{azul}\textbf{[F]}} Una variable ligada es aquella que no cae dentro del alcance de un cuantificador.
            \\ { \color{azul} Las variables ligadas son aquellas ocurrencias de variables que se dan dentro del alcance del cuantificador que la nombra.. }


        \item {\color{azul}\textbf{[V]}} Clausurar significa convertir nuestra fórmula en una oración, ligando todas las variables libres.


        \item {\color{azul}\textbf{[V]}} Una estructura $M$ consta de un dominio y una interpretación que le da sentido a todos nuestros símbolos no lógicos de nuestra función.


        \item {\color{azul}\textbf{[V]}} Si nuestra fórmula tiene variables libres su verdad dependerá de una asignación $s$.


        \item {\color{azul}\textbf{[V]}} Una fórmula es válida si es verdadera en toda estructura (y toda asignación).


        \item {\color{azul}\textbf{[V]}} Si $\varphi$ es válida, entonces $\lnot \varphi$ es insatisfacible.


        \item {\color{azul}\textbf{[V]}} Dos estructuras pueden tener el mismo dominio, pero ser distintas si se interpretan de forma diferente algún símbolo no lógico.


        \item {\color{azul}\textbf{[V]}} Si $\varphi \vDash \theta$ entonces toda estructura que satisface a $\varphi$ satisface $\theta$.


        \item {\color{azul}\textbf{[F]}} La interpretación de un símbolo de función $f$ en una estructura está determinada sólo por el dominio.
        \\ { \color{azul} La interpretación de una función también depende de cómo se asigna una operación específica dentro de ese dominio. }
    \end{enumerate}


    \vspace{\Pspace}
    \item Traducir los siguientes argumentos en lenguaje natural a LPO.
    \\ \textbf{Ejemplos}
    \begin{enumerate}[1.]
        \item Cualquiera puede cocinar.
        \begin{itemize}
            \item $D = \{$Personas$\}$
            \item $P(x) = $ ``$x$ puede cocinar"
        \end{itemize}
        \hspace*{2em} $\forall x P(x)$

        \item Todo el que conoce a Julia la ama.
        \begin{itemize}
            \item $D = \{$Personas$\}$
            \item $j = $ Julia
            \item $Conoce(x,y) = $ ``$x$ conoce a $y$"
            \item $Ama(x,y) = $ ``$x$ ama a $y$"
        \end{itemize}
        \hspace*{2em} $\forall x (Conoce(x,j) \rightarrow Ama(x,j))$
    \end{enumerate}
    
    \textbf{Ejercicios}
    \begin{enumerate}[1.]
        \item No hay barbero que afeite precisamente a aquellos hombres que no se afeitan a sí mismos.
        % - Respuesta
        { \color{azul} 
            \vspace{-0.5cm}
            \begin{itemize}
                \item $D = \{$ Personas $\}$
                \item $B(x) = $ ``$x$ es barbero"
                \item $H(x) = $ ``$x$ es hombre"
                \item $A(x,y) = $ ``$x$ afeita a $y$"
            \end{itemize}
            $\lnot \exists x (B(x) \land \forall y (H(y) \rightarrow (A(x,y) \leftrightarrow \lnot A(y,y))))$
        }


        \vspace{\Pspace}
        \item Andrés odia a aquellas personas que no se odian a sí mismas.
        % - Respuesta
        { \color{azul} 
            \begin{itemize}
                \item $D = \{$ Personas $\}$
                \item $O(x,y) = $ ``$x$ odia a $y$"
                \item $a = $ Andrés
            \end{itemize}
            $\forall x (O(a, x) \rightarrow \lnot O(x,x))$
        }


        \newpage
        \item Nadie en la clase de estadística es más inteligente que cualquiera en la clase de lógica.
        % - Respuesta
        { \color{azul} 
            \begin{itemize}
                \item $D = \{$ Estudiantes $\}$
                \item $E(x) \coloneqq$ ``$x$ está en clase de estadística"
                \item $L(x) \coloneqq$ ``$x$ está en clase de lógica"
                \item $I(x,y) \coloneqq$ ``$x$ es más inteligente que $y$"
            \end{itemize}
            $\lnot \exists x (E(x) \land \forall y (L(y) \rightarrow I(x,y)))$
        }


        \vspace{\Pspace}
        \item Si cualquiera puede resolver el problema, entonces Amaya puede.
        % - Respuesta
        { \color{azul} 
            \begin{itemize}
                \item $D = \{$ Personas $\}$
                \item $R(x) = $ ``$x$ puede resolver el problema"
                \item $a = Amaya$
            \end{itemize}
            $\forall x (R(x) \rightarrow R(a))$
        }


        \vspace{\Pspace}
        \item Nadie ama a un americanista.
        { \color{azul} 
            \begin{itemize}
                \item $D = \{$ Personas $\}$
                \item $A(x,y) = $ ``$x$ ama a $y$"
                \item $M(x) = $ ``$x$ es americanista"
            \end{itemize}
            $\lnot \exists x \exists y (M(y) \land A(x,y)))$
        }


        \vspace{\Pspace}
        \item No existe ningún conjunto que pertenezca precisamente a aquellos conjuntos que no pertenecen a sí mismos.
        { \color{azul}
            \begin{itemize}
                \item $D = \{$ Conjuntos $\}$
                \item $P(x,y) \coloneqq$ ``$x$ pertenece a $y$"
            \end{itemize}
            $\lnot \exists x \forall y (P(x,y) \leftrightarrow \lnot P(y,y))$
        }


        \vspace{\Pspace}
        \item Existen al menos dos personas distintas que son profesores
        { \color{azul}
            \begin{itemize}
                \item $D = \{$ Personas $\}$
                \item $P(x) \coloneqq$ ``$x$ es profesor"
            \end{itemize}
            $\exists x \exists y (x \neq y \land P(x) \land P(y))$
        }


        \vspace{\Pspace}
        \item Existe alguien que conoce a Julia y no la ama.
        { \color{azul}
            \begin{itemize}
                \item $D = \{$ Personas $\}$
                \item $C(x,y) \coloneqq$ ``$x$ conoce a $y$"
                \item $A(x,y) \coloneqq$ ``$x$ ama a $y$"
                \item $j = $ Julia
            \end{itemize}
            $\exists x (C(x, j) \land \lnot A(x, j))$
        }
    \end{enumerate}



    \newpage
    \item Traduzca de LPO a lenguaje natural lo siguiente.
    \begin{enumerate}[A.]
        \item Estructura 1:
        \begin{itemize}
            \item $D = $Personas
            \item $Conoce(x,y) = $ ``$x$ conoce a $y$"
            \item $Odia(x,y) = $ ``$x$ odia a $y$"
            \item $j = $ Juan
            \item $g = $ Gerardo
        \end{itemize}
        \textbf{Ejemplo}
        \\ $\forall x (Conoce(x,j) \rightarrow Odia(x, j))$
        \\ ``Todo el que conoce a Juan lo odia"
        \\ \textbf{Ejercicios}
        \begin{enumerate}[1.]
            \item $\lnot \exists x Odia(x, g)$
            \\ { \color{azul} ``Nadie odia a Gerardo" }


            \item $\exists x \lnot Conoce(x, j)$
            \\ { \color{azul} ``Alguien no conoce a Juan" }


            \item $\forall x (Odia(x,j) \leftrightarrow Odia(x,g))$
            \\ { \color{azul} ``Toda persona que odia a Juan también odia a Gerardo y toda persona que odia a Gerardo también odia a Juan" }
        \end{enumerate}

        
        \vspace{\Pspace}
        \item Estructura 2:
        \begin{itemize}
            \item $D = ${Personas, cursos}
            \item $Est(x) \coloneqq$ ``$x$ es un estudiante"
            \item $Prof(x) \coloneqq$ ``$x$ es un profesor"
            \item $Curso(x) \coloneqq$ ``$x$ es un curso"
            \item $Toma(x,y) = $ ``$x$ toma el curso $y$"
            \item $Conoce(x,y) = $ ``$x$ conoce a $y$"
            \item log = Lógica
            \item mat = Matemáticas
        \end{itemize}
        \begin{enumerate}[1.]
            \item $\forall x (Est(x) \rightarrow \exists y (Curso(y) \land Toma(x,y)))$
            \\ { \color{azul} ``Toda persona que es estudiante toma un curso" }


            \item $\forall x (Est(x) \rightarrow \lnot Toma(x,mat))$
            \\ { \color{azul} ``Toda persona que es estudiante no toma el curso de matemáticas" }
        \end{enumerate}
    \end{enumerate}



    \item Para las siguientes fbf, determine:
    \\ a) El alcance de cada cuantificador.
    \\ b) Variables libres (FV()) y ligadas.
    \\ c) Si hay variables libres, clausurar de manera existencial y universal.
    \begin{enumerate}[1.]
        \item $\forall x (P(x) \rightarrow Q(x))$ \par
        { \color{azul}
            \begin{tabular}{c|c|c}
                \textbf{Alcance} & $FV(\varphi)$ & $LV(\varphi)$\\
                \hline
                $\forall x$: $P(x) \rightarrow Q(x)$ & $\emptyset$ & $x$
            \end{tabular} \\
        }

        \vspace{\Pspace}
        \item $\exists y P(y,z)$ \par
        { \color{azul}
            \begin{tabular}{c|c|c|c|c}
                \textbf{Alcance} & $FV(\varphi)$ & $LV(\varphi)$ & $Cl_{\exists}$ & $Cl_{\forall}$\\
                \hline
                $\exists y$: $P(y,z)$ & $z$ & $y$ & $\exists z \exists y P(y,z)$ & $\forall z \exists y P(y,z)$
            \end{tabular} \\
        }


        \vspace{\Pspace}
        \item $\forall x (P(x) \land \forall y Q(y))$ \par
        { \color{azul}
            \begin{tabular}{c|c|c}
                \textbf{Alcance} & $FV(\varphi)$ & $LV(\varphi)$\\
                \hline
                $\forall x$: $P(x) \land \forall y Q(y)$, $\forall y$: $Q(y)$ & $\emptyset$ & $x$, $y$
            \end{tabular} \\
        }


        \vspace{\Pspace}
        \item $\forall x (P(y) \rightarrow \exists y (Q(y) \land R(x,y)))$ \par
        { \color{azul}
            \begin{tabular}{c|c|c}
                \textbf{Alcance} & $FV(\varphi)$ & $LV(\varphi)$\\
                \hline
                $\forall x$: $P(y) \rightarrow \exists y (Q(y) \land R(x,y))$, $\exists y$: $Q(y) \land R(x,y)$ & $\emptyset$ & $x$, $y$
            \end{tabular} \\
        }


        \vspace{\Pspace}
        \item $\exists x \forall y (R(x,y) \rightarrow R(y,x))$ \par
        { \color{azul}
            \begin{tabular}{c|c|c}
                \textbf{Alcance} & $FV(\varphi)$ & $LV(\varphi)$\\
                \hline
                $\exists x$: $R(x,y) \rightarrow R(y,x)$, $\forall y$: $R(x,y) \rightarrow R(y,x)$ & $\emptyset$ & $x$, $y$
            \end{tabular} \\
        }


        \vspace{\Pspace}
        \item $\forall x (P(f(x)) \lor \exists y Q(g(y), x))$ \par
        { \color{azul}
            \begin{tabular}{c|c|c|c|c}
                \textbf{Alcance} & $FV(\varphi)$ & $LV(\varphi)$ & $Cl_{\exists}$ & $Cl_{\forall}$\\
                \hline
            \end{tabular} \\
        }


        \vspace{\Pspace}
        \item $\exists z (\forall x P(x,z) \rightarrow \exists x Q(x,z))$ \par
        { \color{azul}
            \begin{tabular}{c|c|c|c|c}
                \textbf{Alcance} & $FV(\varphi)$ & $LV(\varphi)$ & $Cl_{\exists}$ & $Cl_{\forall}$\\
                \hline
            \end{tabular} \\
        }


        \vspace{\Pspace}
        \item $\forall x P(x, f(y))$ \par
        { \color{azul}
            \begin{tabular}{c|c|c|c|c}
                \textbf{Alcance} & $FV(\varphi)$ & $LV(\varphi)$ & $Cl_{\exists}$ & $Cl_{\forall}$\\
                \hline
                $\forall x$: $P(x, f(y))$ & $y$ & $x$ & $\exists y (\forall x P(x, f(y)))$ & $\forall y (\forall x P(x, f(y)))$
            \end{tabular} \\
        }


        \vspace{\Pspace}
        \item $\exists x (Q(x) \land \forall y (x = y \rightarrow P(y)))$ \par
        { \color{azul}
            \begin{tabular}{c|c|c|c|c}
                \textbf{Alcance} & $FV(\varphi)$ & $LV(\varphi)$ & $Cl_{\exists}$ & $Cl_{\forall}$\\
                \hline
            \end{tabular} \\
        }


        \vspace{\Pspace}
        \item $\exists y (R(y) \land \forall x (P(x,y) \rightarrow \exists y Q(x,y)))$ \par
        { \color{azul}
            \begin{tabular}{c|c|c|c|c}
                \textbf{Alcance} & $FV(\varphi)$ & $LV(\varphi)$ & $Cl_{\exists}$ & $Cl_{\forall}$\\
                \hline
            \end{tabular} \\
        }
    \end{enumerate}



    \item Para las siguientes fbf
    \\ a) Proponga una estructura $M$ que la satisfaga
    \\ b) Proponga una estructura $M$ que la haga falsa (contra-modelo)
    \\ \textbf{Ejemplos}
    \\ $\forall x (P(x) \rightarrow \exists y R(x,y))$
    \begin{enumerate}[1.]
        \item Propuesta de modelo 1.
        \begin{itemize}
            \item $D = \{0,1\}$
            \item $P^{M} = \{0\}$
            \item $R^{M} = \{(0,0)\}$
        \end{itemize}

        \item Propuesta de modelo 2.
        \begin{itemize}
            \item $D = \mathbb{N}$
            \item $P^{M}(x) \coloneqq $ ``$x$ es par"
            \item $R^{M}(x,y) \coloneqq$ ``$x$ es mayor que $y$"
        \end{itemize}

        \item Propuesta de contra-modelo 1.
        \begin{itemize}
            \item $D = \{a\}$
            \item $P^{M} = \{a\}$
            \item $R^{M} = \emptyset$ (es falsa porque hay $a$ con $P(a)$ pero sin testigo $y$).
        \end{itemize}

        \item Propuesta de contra-modelo 2.
        \begin{itemize}
            \item $D = \mathbb{N}$
            \item $P^{M}(x) \coloneqq $ ``$x$ es un número natural"
            \item $R^{M}(x,y) \coloneqq $ ``$x$ es menor que $y$"
        \end{itemize}
    \end{enumerate}


    \par \textbf{Ejercicios}
    \begin{enumerate}[1.]
        \item $\exists x (P(x) \land \forall y Q(y))$
        \par a) $M$:
        { \color{azul} 
            \begin{itemize}
                \item $D = $
            \end{itemize}
        }

        \par b) $M$:
        { \color{azul} 
            \begin{itemize}
                \item $D = $
            \end{itemize}
        }
        
        
        \vspace{\Pspace}
        \item $\forall x \exists y (x = y \land R(x,y))$
        % - Respuestas:
        \par a) $M$:
        { \color{azul} 
            \begin{itemize}
                \item $D = \mathbb{N}$
                \item $R^{M}(x,y) \coloneqq $ ``$\frac{x}{y} = 1$"
            \end{itemize}
        }

        \par b) $M$:
        { \color{azul} 
            \begin{itemize}
                \item $D = \mathbb{N}$
                \item $R^{M}(x,y) \coloneqq $ ``$x > y$"
            \end{itemize}
        }

        \vspace{\Pspace}
        \item $\forall x P(f(x))$
        \par a) $M$:
        { \color{azul} 
            \begin{itemize}
                \item $D = \mathbb{N}$
                \item $P^{M}(x) \coloneqq$ ``$x$ es par"
                \item $f^{M}(x) = 2x$
            \end{itemize}
        }

        \par b) $M$:
        { \color{azul} 
            \begin{itemize}
                \item $D = \mathbb{N}$
                \item $P^{M}(x) \coloneqq$ ``$x$ es par"
                \item $f^{M}(x) = 2x + 1$
            \end{itemize}
        }


        \vspace{\Pspace}
        \item $\exists x (x = c \land f(f(x)) = x)$
        \par a) $M$:
        { \color{azul} 
            \begin{itemize}
                \item $D = \mathbb{R}$
                \item $c^{M} = 5$
                \item $f^{M}(x) = 1(x)$
            \end{itemize}
        }

        \par b) $M$:
        { \color{azul} 
            \begin{itemize}
                \item $D = \mathbb{R}$
                \item $c^{M} = 5$
                \item $f^{M}(x) = x + 1$
            \end{itemize}
        }


        \vspace{\Pspace}
        \item $\forall x (f(x) = x) \rightarrow \forall x P(x)$
        \par a) $M$:
        { \color{azul} 
            \begin{itemize}
                \item $D = $
            \end{itemize}
        }

        \par b) $M$:
        { \color{azul} 
            \begin{itemize}
                \item $D = $
            \end{itemize}
        }


        \vspace{\Pspace}
        \item $\forall x \forall y (E(x,y) \rightarrow E(y,x))$
        \par a) $M$:
        { \color{azul} 
            \begin{itemize}
                \item $D = \{$ (Fer, Fer), (Fer, Javi), (Javi, Fer), (Javi, Javi) $\}$
                \item $E^{M}(x, y) \coloneqq $ ``$x$ conoce a $y$"
            \end{itemize}
        }

        \par b) $M$:
        { \color{azul} 
            \begin{itemize}
                \item $D = \{$ (Fer, Javi) $\}$
                \item $E^{M}(x, y) \coloneqq $ ``$x$ conoce a $y$"
            \end{itemize}
        }
    \end{enumerate}

    
    
    \newpage
    \item Dada la siguiente estructura $M$, determine si $M \vDash \varphi_{n}$
    \begin{itemize}
        \item $D = \mathbb{R}$
        \item $a^{M} = 4$
        \item $f^{M}(n) = n - 2$
        \item $Par^{M}(n) \coloneqq $ ``$n$ es par"
        \item $R^{M}(m,n) \coloneqq m < n$
        \item Secuencia: $s(x) = 3$, $s(y) = 2$
    \end{itemize}

    \par \textbf{Ejemplo}
    \\ $\varphi = R(x, f(y))$
    \\ $\rightarrow s^{*}(x) = 3, s^{*}(y) = 2, s^{*}(f(y)) = 0$
    \\ $\rightarrow R(x, f(y)) = R(3,0); (3,0) \notin R^{M} \therefore M \nvDash \varphi_{1}$

    \vspace{\Pspace}
    \par \textbf{Ejercicios}
    \vspace{-0.2cm}
    \begin{enumerate}[1.]
        \item $\varphi_{1} = Par(f(x))$
        % - Respuesta:
        { \color{azul}
            \\ $\rightarrow s^{*}(x) = 3$ 
            \\ $\rightarrow s^{*}(f(x)) = 3 - 2 = 1$; $1 \notin Par(x)$
            \\ $\rightarrow$ $\therefore M \nvDash \varphi_{1}$
        }


        \item $\varphi_{2} = \forall x R(x, f(x))$
        % - Respuesta:
        { \color{azul}
            \\ $\rightarrow \forall d \in \mathbb{R}$ 
            \\ $\rightarrow s^{*}[x|d](x) = d$
            \\ $\rightarrow s^{*}[x|d](f(x)) = d - 2$
            \\ $\rightarrow R(d, d-2) \notin R^{M}$ ya que $d \nless d - 2$
            \\ $\rightarrow$ $\therefore M \nvDash \varphi_{2}$
        }


        \item $\varphi_{3} = \forall x (Par(x) \rightarrow Par(f(x))$
        % - Respuesta:
        { \color{azul}
            \\ $\rightarrow \forall d \in \mathbb{R}$ 
            \\ $\rightarrow s^{*}[x|d](x) = d$
            \\ $\rightarrow s^{*}[x|d](f(x)) = d - 2$ 
            \\ $\rightarrow d \in Par^{M}$ solo si $d \in \{x \in \mathbb{Z} | x = 2k, k \in \mathbb{Z} \}$
            \\ $\rightarrow$ Reemplazamos $d$ por $2k$ en $d - 2$ 
            \\ $\rightarrow 2k - 2 = 2(k - 1) = 2j$; $d - 2 \in Par^{M}$
            \\ $\rightarrow$ $\therefore M \vDash \varphi_{3}$
        }


        \item $\varphi_{4} = \forall x R(a, x)$
        % - Respuesta:
        { \color{azul}
            \\ $\rightarrow \forall d \in \mathbb{R}$ 
            \\ $\rightarrow s^{*}[x|d] = d$
            \\ $\rightarrow R(4,d) = 4 < d$ 
            \\ $\rightarrow$ Dado que la fórmula no se cumple con valores para $d$ menores o iguales a 4 la fórmula no es satisfacible.
            \\ $\rightarrow$ $\therefore M \nvDash \varphi_{4}$ 
        }


        \item $\varphi_{5} = \exists y R(a, f(y))$
        % - Respuesta:
        { \color{azul}
            \\ $\rightarrow \forall d \in \mathbb{R}$
            \\ $\rightarrow s^{*}[y|d](y) = d$
            \\ $\rightarrow s^{*}[y|d](f(y)) = d - 2$
            \\ $\rightarrow R(4, d - 2) = 4 < d - 2$
            \\ $\rightarrow R(4, d - 2)$ se cumple cuando $d > 6$, por lo tanto si existen valores que hacen la fórmula satisfacible.
            \\ $\rightarrow$ $\therefore M \vDash \varphi_{5}$
        }
    \end{enumerate}

\end{enumerate}
\end{document}
