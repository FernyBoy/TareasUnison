\documentclass[a4paper, 11pt]{article}

% -- Language --

% -- Coments
\usepackage{verbatim}

% ----- Fonts -----
% -- Fuente --
% \usepackage{fontspec}
% \setmonofont{JetBrainsMono Nerd Font}  

% -- Color --
\usepackage{xcolor}
%\definecolor{azul}{RGB}{00,33,99}
\definecolor{azul}{RGB}{35,72,180}

% -- Page Margin --
\usepackage[margin=1in]{geometry}

% -- Espaciados --
\newcommand{\Pspace}{0.5cm}
\newcommand{\Aspace}{0.2cm}

% -- Columnas --
\usepackage{multicol}

% -- Imagenes --
\usepackage{graphicx}
\usepackage{float}

% -- Matemáticas --
\usepackage{amsmath, amssymb}

% -- Gráficas --
\usepackage{pgfplots}
\pgfplotsset{compat=1.18}

% -- Código --
\usepackage{listings}
\lstset{
    language=C++,                   % Lenguaje del código
    basicstyle=\ttfamily\small,     % Fuente del código
    keywordstyle=\color{blue},      % Color de palabras clave
    commentstyle=\color{gray},      % Color de comentarios
    stringstyle=\color{red},        % Color de cadenas
    numbers=left,                   % Números de línea a la izquierda
    numberstyle=\tiny\color{gray},
    breaklines=true,                % Permitir saltos de línea
    frame=single                    % Marco alrededor del código
}

\usepackage{enumerate}

\begin{comment}
-- Formato de pregunta simple
    % - Problema n
    \vspace{\Pspace}
    \item Problema \par
        % Respuesta:
        \vspace{\Aspace} \par
        { \color{azul}  }


 -- Formato de pregunta multimple
        % - Problema 
        \vspace{\Pspace}
        \item  \par
             % Respuestas:
            \vspace{\Aspace} \par
            a) 
            \\ { \color{azul} }

            \vspace{\Aspace} \par
            b) 
            \\ { \color{azul} }


 -- Formato de imagen
\begin{figure}[!ht]
    \centering
    \includegraphics[width=0.5\textwidth]{}
\end{figure}


 -- Formato de tabla
\begin{tabular}{c|c}
    \textbf{A} & \textbf{B} \\
    \hline
    Texto & Texto \\
    Texto & Texto
\end{tabular} \\
\end{comment}


\title
{
    Análisis Lógico 2025-2 \\
    Tarea pre-examen Unidad II
}

    \begin{document}

    \maketitle

    \begin{center}
        \begin{tabular}{r|l}
            \textbf{Expediente} & \textbf{Nombre} \\ \hline
            219208106 & Bórquez Guerrero Angel Fernando \\
        \end{tabular}
    \end{center}

    \rule{\linewidth}{0.3mm}
    
    
\begin{enumerate}[I.]
    \item Responde si las siguientes afirmaciones son verdaderas (V) o falsas (F), justificando en una línea si es falso (corrige la afirmación)
    \begin{enumerate}[1.]
        \item {\color{azul}\textbf{[V]}} En un lenguaje de primer orden no puede haber símbolos de función de aridad 0.


        \item {\color{azul}\textbf{[V]}} Toda constante es, desde el punto de vista sintáctico, un término.


        \item {\color{azul}\textbf{[F]}} El alcance de un cuantificador $\forall x$ o $\exists x$ es toda la fórmula en la que aparece.
        \\ { \color{azul} El alcance se limita a la subfórmula que sigue al cuantificador. }


        \item {\color{azul}\textbf{[F]}} El alfabeto de primer orden se compone de variables, constantes, funciones, predicados, conectivos lógicos, símbolos especiales.
        \\ { \color{azul} Faltan los cuantificadores. }


        \item {\color{azul}\textbf{[F]}} Los predicados se representan con una letra minúscula (Ej. $p(x) = $ ``$x$ es par").
        \\ { \color{azul} La letra debe de ser mayúscula. }


        \item {\color{azul}\textbf{[F]}} La aridad representa el número de predicado o función sobre el cual estamos hablando.
        \\ { \color{azul} La aridad es el número de elementos (objetos o variables) sobre los que un predicado o una función se aplica. }


        \item {\color{azul}\textbf{[V]}} $Ama(x,y) = $ ``$x$ ama a $y$" es un ejemplo de aridad 2.


        \item {\color{azul}\textbf{[F]}} Un término es cualquier expresión que ``nombra" un objeto del dominio. Nos dice si algo es verdadero o falso.
        \\ { \color{azul} Aunque es verdad que un término nombra un objeto del dominio, lo que nos dice si algo es verdadero o falso es un predicado. }


        \item {\color{azul}\textbf{[F]}} Un predicado es un ejemplo de término.
        \\ { \color{azul} Un predicado dice si algo es verdadero o falso, los términos representan objetos del dominio.  }


        \item {\color{azul}\textbf{[V]}} Una fórmula atómica (o átomo) es un predicado aplicado a un término.


        \item {\color{azul}\textbf{[F]}} LPO significa: Lógica de primer orden.
        \\ { \color{azul} LPO significa: Lenguaje de Primer Orden. }


        \item {\color{azul}\textbf{[F]}} Una variable ligada es aquella que no cae dentro del alcance de un cuantificador.
            \\ { \color{azul} Las variables ligadas son aquellas ocurrencias de variables que se dan dentro del alcance del cuantificador que la nombra.. }


        \item {\color{azul}\textbf{[V]}} Clausurar significa convertir nuestra fórmula en una oración, ligando todas las variables libres.


        \item {\color{azul}\textbf{[V]}} Una estructura $M$ consta de un dominio y una interpretación que le da sentido a todos nuestros símbolos no lógicos de nuestra función.


        \item {\color{azul}\textbf{[V]}} Si nuestra fórmula tiene variables libres su verdad dependerá de una asignación $s$.


        \item {\color{azul}\textbf{[V]}} Una fórmula es válida si es verdadera en toda estructura (y toda asignación).


        \item {\color{azul}\textbf{[V]}} Si $\varphi$ es válida, entonces $\lnot \varphi$ es insatisfacible.


        \item {\color{azul}\textbf{[V]}} Dos estructuras pueden tener el mismo dominio, pero ser distintas si se interpretan de forma diferente algún símbolo no lógico.


        \item {\color{azul}\textbf{[V]}} Si $\varphi \vDash \theta$ entonces toda estructura que satisface a $\varphi$ satisface $\theta$.


        \item {\color{azul}\textbf{[F]}} La interpretación de un símbolo de función $f$ en una estructura está determinada sólo por el dominio.
        \\ { \color{azul} La interpretación de una función también depende de cómo se asigna una operación específica dentro de ese dominio. }
    \end{enumerate}



    \item Traducir los siguientes argumentos en lenguaje natural a LPO.
    \\ \textbf{Ejemplos}
    \begin{enumerate}[1.]
        \item Cualquiera puede cocinar.
        \begin{itemize}
            \item $D = \{$Personas$\}$
            \item $P(x) = $ ``$x$ puede cocinar"
        \end{itemize}
        \hspace*{2em} $\forall x P(x)$

        \item Todo el que conoce a Julia la ama.
        \begin{itemize}
            \item $D = \{$Personas$\}$
            \item $j = $ Julia
            \item $Conoce(x,y) = $ ``$x$ conoce a $y$"
            \item $Ama(x,y) = $ ``$x$ ama a $y$"
        \end{itemize}
        \hspace*{2em} $\forall x (Conoce(x,j) \rightarrow Ama(x,j))$
    \end{enumerate}
    \par \textbf{Ejercicios}
    \begin{enumerate}[1.]
        \item No hay barbero que afeite precisamente a aquellos hombres que no se afeitan a sí mismos.


        \item Andrés odia a aquellas personas que no sea odian a sí mismas.


        \item Nadie en la clase de estadística es más inteligente que cualquiera en la clase de lógica.


        \item Si cualquiera puede resolver el problema, entonces Amaya puede.


        \item Nadie ama a un americanista.


        \item No existe ningún conjunto que pertenezca precisamente a aquellos conjuntos que no pertenecen a sí mismos.


        \item Existen al menos dos personas distintas que son profesores.


        \item Existe alguien que conoce a Julia y no la ama.
    \end{enumerate}



    \item Traduzca de LPO a lenguaje natural lo siguiente.
    \begin{enumerate}[A.]
        \item Estructura 1:
        \begin{itemize}
            \item $D = $Personas
            \item $Conoce(x,y) = $ ``$x$ conoce a $y$"
            \item $Odia(x,y) = $ ``$x$ odia a $y$"
            \item $j = $ Juan
            \item $g = $ Gerardo
        \end{itemize}
        \textbf{Ejemplo}
        \\ $\forall x (Conoce(x,j) \rightarrow Odia(x, j))$
        \\ ``Todo el que conoce a Juan lo odia"
        \\ \textbf{Ejercicios}
        \begin{enumerate}[1.]
            \item $\lnot \exists x Odia(x, g)$


            \item $\exists x \lnot Conoce(x, j)$


            \item $\forall x (Odia(x,j) \leftrightarrow Odia(x,g))$
        \end{enumerate}

        
        \vspace{\Pspace}
        \item Estructura 2:
        \begin{itemize}
            \item $D = ${Personas, cursos}
            \item $Est(x) = $ ``$x$ es un estudiante"
            \item $Prof(x) = $ ``$x$ ees un profesor"
            \item $Toma(x,y) = $ ``$x$ toma el curso $y$"
            \item $Conoce(x,y) = $ ``$x$ conoce a $y$"
            \item log = Lógica
            \item mat = Matemáticas
        \end{itemize}
        \begin{enumerate}[1.]
            \item $\forall x (Est(x) \rightarrow \exists y (Curso(y) \land Toma(x,y)))$


            \item $\exists x (Prof(x) \land \forall y ((Curso(y) \land Toma(y, log)) \rightarrow Conoce(x,y)))$


            \item $\forall x (Est(x) \rightarrow \lnot Toma(x,mat))$
        \end{enumerate}
    \end{enumerate}



    \item Para las siguientes fbf, determine:
    \\ a) El alcance de cada cuantificador.
    \\ b) Variables libres (FV()) y ligadas.
    \\ c) Si hay variables libres, clausurar de manera existencial y universa.
    \begin{enumerate}[1.]
        \item $\forall x (P(x) \rightarrow Q(x))$


        \item $\exists y P(y,z)$


        \item $\forall x (P(x) \land \forall y Q(y))$


        \item $\forall x (P(y) \rightarrow \exists y (Q(y) \land R(x,y)))$


        \item $\exists x \forall y (R(x,y) \rightarrow R(y,x))$


        \item $\forall x (P(f(x)) \lor \exists y Q(g(y), x))$


        \item $\exists z (\forall x P(x,z) \rightarrow \exists x Q(x,z))$


        \item $\forall x P(x, f(y))$


        \item $\exists x (Q(x) \land \forall y (x = y \rightarrow P(y)))$


        \item $\exists y (R(y) \land \forall x (P(x,y) \rightarrow \exists y Q(x,y)))$
    \end{enumerate}



\end{enumerate}
\end{document}
