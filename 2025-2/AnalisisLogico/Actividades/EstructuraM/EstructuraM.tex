\documentclass[a4paper, 11pt]{article}

% -- Language --
\usepackage[spanish]{babel}
\usepackage[utf8]{inputenc}

% -- Coments
\usepackage{verbatim}

% ----- Fonts -----
% -- Fuente --
% \usepackage{fontspec}
% \setmonofont{JetBrainsMono Nerd Font}  

% -- Color --
\usepackage{xcolor}
%\definecolor{azul}{RGB}{00,33,99}
\definecolor{azul}{RGB}{35,72,180}

% -- Page Margin --
\usepackage[margin=1in]{geometry}

% -- Espaciados --
\newcommand{\Pspace}{0.5cm}
\newcommand{\Aspace}{0.2cm}

% -- Columnas --
\usepackage{multicol}

% -- Imagenes --
\usepackage{graphicx}
\usepackage{float}

% -- Matemáticas --
\usepackage{amsmath, amssymb}
\usepackage{mathtools}

% -- Gráficas --
\usepackage{pgfplots}
\pgfplotsset{compat=1.18}

% -- Código --
\usepackage{listings}
\lstset{
    language=C++,                   % Lenguaje del código
    basicstyle=\ttfamily\small,     % Fuente del código
    keywordstyle=\color{blue},      % Color de palabras clave
    commentstyle=\color{gray},      % Color de comentarios
    stringstyle=\color{red},        % Color de cadenas
    numbers=left,                   % Números de línea a la izquierda
    numberstyle=\tiny\color{gray},
    breaklines=true,                % Permitir saltos de línea
    frame=single                    % Marco alrededor del código
}

\begin{comment}
-- Formato de pregunta simple
    % - Problema n
    \vspace{\Pspace}
    \item Problema \par
        % Respuesta:
        \vspace{\Aspace} \par
        { \color{azul}  }


 -- Formato de pregunta multimple
        % - Problema 
        \vspace{\Pspace}
        \item  \par
             % Respuestas:
            \vspace{\Aspace} \par
            a) 
            \\ { \color{azul} }

            \vspace{\Aspace} \par
            b) 
            \\ { \color{azul} }


 -- Formato de imagen
\begin{figure}[!ht]
    \centering
    \includegraphics[width=0.5\textwidth]{}
\end{figure}


 -- Formato de tabla
\begin{tabular}{c|c}
    \textbf{A} & \textbf{B} \\
    \hline
    Texto & Texto \\
    Texto & Texto
\end{tabular} \\
\end{comment}


\title
{
    Análisis Lógico 2025-2 \\
    Estructura M, validez y satisfacibilidad
}

    \begin{document}

    \maketitle

    \begin{center}
        \begin{tabular}{r|l}
            \textbf{Expediente} & \textbf{Nombre} \\ \hline
            219208106 & Bórquez Guerrero Angel Fernando \\
        \end{tabular}
    \end{center}

    \rule{\linewidth}{0.3mm}

    \vspace{0.3cm}
    \begin{enumerate}
        \vspace{\Pspace}
        \item Elegir una estructura $M$ que haga válidos, satisfacibles e inválidos a las siguientes fbf.\par
             % Respuestas:
            \vspace{\Aspace} \par
            a) $\forall x \exists y R(x,y)$
            \\ { \color{azul}  
                Válido
                \par $M: D = \mathbb{N}, R^{M}(m,n) = m = n$
                \par $\rightarrow \forall d \in D$
                \par $\rightarrow s^{*}[x|d](x) = d$
                \par $\rightarrow \exists e \in D$
                \par $\rightarrow s^{*}[y|e](y) = d$
                \par $\rightarrow d \in R^{M} \forall d$
                \par $\rightarrow d \in R^{M} \exists e$
                \par $\rightarrow \; \therefore M \vDash \varphi$

                \vspace{\Aspace}
                Satisfacible
                \par $M: D = \mathbb{N}, R^{M}(m,n) = m > n$
                \par En este caso podemos ver que se cumple para todos los números naturales, ya que siempre hay una $y$ menor a $x$, excepto cuando llegamos al 0 ya que no existe un número que sea menor, por lo tanto la estructura es satisfacible.
                \par $\; \therefore M \vDash \varphi$

                \vspace{\Aspace}
                Insatisfacible
                \par $M: D = \{ 256, 256, 256 \}, R^{M}(m,n) = m > n$
                \par Dado que todos los elementos del dominio son el mismo número entonces no existse una $R(x,y) \in R^{M}$, esto nos dice que esta estructura es inválida.
                \par $\; \therefore M \nvDash \varphi$
            }



            \newpage
            b) $\exists y \forall x R(x,y)$
            \\ { \color{azul} 
                Válido
                \par $M: D = \{ (Fer, Fer), (Flor, Flor), (Fer, Flor), (Flor, Fer) \}, R^{M}(x,y) \coloneqq$ ``$x$ conoce a $y$''
                La estructura nos dice que que existe una $y$ que conoce a todas las $x$, y dado que sin importar los valores que tomen las variables $x$ y $y$ el predicado $R^{M}$ se cumple para todos los casos podemos decir que esta es una estructura válida.
                \par $\; \therefore M \vDash \varphi$

                \vspace{\Aspace}
                Satisfacible
                \par $M: D = \{ (Fer, Fer), (Flor, Flor), (Fer, Flor) \}, R^{M}(x,y) \coloneqq$ ``$x$ conoce a $y$''
                \par Esta estructura no se cumple para todos los casos, ya que si adoptamos el valor de $x = Fer$ este conocerá a los demás valores que adopte $y$, pero si tomamamos los valores $x = Flor$ y $y = Fer$ entonces $R(x,y) \notin R^{M}$, ya que contamos con un caso que no cumple la estructura pero otro que sí cumple con ella podemos decir que la estructura es satisfacible.
                \par $\; \therefore M \vDash \varphi$

                \vspace{\Aspace}
                Insatisfacible
                \par $M: D = \{ (Fer, Fer), (Flor, Flor) \}, R^{M}(x,y) \coloneqq$ ``$x$ conoce a $y$''
                \par Si adoptamos el caso donde $x = Fer$ no podremos cumplir el caso $\exists y \forall x R(x,y)$ ya que Fer no conoce a Flor, y si adoptamos el caso donde $x = Flor$ tampoco podremos cumplir el caso ya que Flor no conoce a Fer. Ya que no existe un caso que cumpla $\exists y \forall x R(x,y)$ la estructura no es válida.
                \par $\; \therefore M \nvDash \varphi$
            }



            \vspace{\Pspace} \par
            c) $\forall x P(x) \rightarrow \exists x P(x)$
            \\ { \color{azul} 
                Utilizando equivalencias podemos desarrollar la fórmula anterior de la siguiente forma:
                \par $\rightarrow \lnot \forall x P(x) \lor \exists x P(x)$
                \par $\rightarrow \exists x \lnot P(x) \lor \exists x P(x)$
                \par $\rightarrow \exists x (\lnot P(x) \lor P(x))$
                \par Dado que $\lnot P(x) \lor P(x)$ es una tautología entonces el enunciado siempre será valido en todo caso.
                \par $\; \therefore M \vDash \varphi$
            }



            \vspace{\Pspace} \par
            d) $\forall x (P(x) \rightarrow \lnot P(x))$
            \\ { \color{azul} 
                Válido
                \par $M: D = \{\, x \in \mathbb{N} \mid x = 2k,\; k \in \mathbb{N} \,\}$, $P^{M}(x) \coloneqq$ ``$x$ es impar''
                \par $\rightarrow \forall d \in D$
                \par $\rightarrow s^{*}[x\, |\, d](x) = d$
                \par $\rightarrow d \notin P^{M} \forall d$
                \par $\rightarrow d \in \lnot P^{M} \forall d$
                \par $\rightarrow$ Podemos traducir esto como $(0 \rightarrow 1) = 1$
                \par $\rightarrow \; \therefore M \vDash \varphi$

                \newpage
                Satisfacible
                \par $M: D = \{ 2, 3 \}$, $P^{M}(x) \coloneqq$ ``$x$ es impar''
                \par $\rightarrow s^{*}(x) = 2$
                \par $\rightarrow 2 \notin P^{M}$
                \par $\rightarrow 2 \in \lnot P^{M}$
                \par $\rightarrow$ Este caso puede ser traducido como $(0 \rightarrow 1) = 1$.
                \par $\rightarrow s^{*}(x) = 3$
                \par $\rightarrow 3 \in P^{M}$
                \par $\rightarrow 3 \notin \lnot P^{M}$
                \par $\rightarrow$ Mientras que este puede ser traducido como $(1 \rightarrow 0) = 0$.
                \par $\rightarrow$ Hemos encontrado un caso en donde la estructura es válida mientras que en el otro caso no lo es, por lo tanto podemos decir que es satisfacible.
                \par $\rightarrow \; \therefore M \vDash \varphi$\

                \vspace{\Aspace}
                Insatisfacible
                \par $M: D = \{\, x \in \mathbb{N} \mid x = 2k + 1,\; k \in \mathbb{N} \,\}$, $P^{M}(x) \coloneqq$ ``$x$ es impar''
                \par $\rightarrow \forall d \in D$
                \par $\rightarrow s^{*}[x\, |\, d](x) = d$
                \par $\rightarrow d \in P^{M} \forall d$
                \par $\rightarrow d \notin \lnot P^{M} \forall d$
                \par $\rightarrow$ Podemos traducir esto como $(1 \rightarrow 0) = 0$
                \par $\rightarrow \; \therefore M \nvDash \varphi$

            }
    \end{enumerate}
\end{document}
