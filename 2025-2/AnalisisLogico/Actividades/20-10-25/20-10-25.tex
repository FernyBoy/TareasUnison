\documentclass[a4paper, 11pt]{article}

% -- Language --
%\usepackage[spanish]{babel}
% \usepackage[utf8]{inputenc}

% -- Coments
\usepackage{verbatim}

% ----- Fonts -----
% -- Fuente --
% \usepackage{fontspec}
% \setmonofont{JetBrainsMono Nerd Font}  

% -- Color --
\usepackage{xcolor}
%\definecolor{azul}{RGB}{00,33,99}
\definecolor{azul}{RGB}{35,72,180}

% -- Page Margin --
\usepackage[margin=1in]{geometry}

% -- Espaciados --
\newcommand{\Pspace}{0.5cm}
\newcommand{\Aspace}{0.2cm}

% -- Columnas --
\usepackage{multicol}

% -- Imagenes --
\usepackage{graphicx}
\usepackage{float}

% -- Matemáticas --
\usepackage{amsmath, amssymb}

% -- Gráficas --
\usepackage{pgfplots}
\pgfplotsset{compat=1.18}

% -- Código --
\usepackage{listings}
\lstset{
    language=C++,                   % Lenguaje del código
    basicstyle=\ttfamily\small,     % Fuente del código
    keywordstyle=\color{blue},      % Color de palabras clave
    commentstyle=\color{gray},      % Color de comentarios
    stringstyle=\color{red},        % Color de cadenas
    numbers=left,                   % Números de línea a la izquierda
    numberstyle=\tiny\color{gray},
    breaklines=true,                % Permitir saltos de línea
    frame=single                    % Marco alrededor del código
}

\begin{comment}
-- Formato de pregunta simple
    % - Problema n
    \vspace{\Pspace}
    \item Problema \par
        % Respuesta:
        \vspace{\Pspace} \par
        { \color{azul}  }


 -- Formato de pregunta multimple
        % - Problema 
        \vspace{\Pspace}
        \item Problema \par
             % Respuestas:
            \vspace{\Pspace} \par
            a) Inciso
            \\ { \color{azul} }

            \vspace{\Pspace} \par
            b) Inciso
            \\ { \color{azul} }


 -- Formato de imagen
\begin{figure}[!ht]
    \centering
    \includegraphics[width=0.5\textwidth]{}
\end{figure}


 -- Formato de tabla
\begin{tabular}{c|c}
    \textbf{A} & \textbf{B} \\
    \hline
    Texto & Texto \\
    Texto & Texto
\end{tabular} \\

            { \color{azul} 
                \begin{itemize}
                    \item $P(x) = $ ``$x$ "
                    \item $H(x) = $ ``$x$ "
                \end{itemize}
                \hspace*{2em} $$
            }
\end{comment}


\title
{
    Análisis Lógico 2025-2 \\
    Actividad 20-10-25
}

    \begin{document}

    \maketitle

    \begin{center}
        \begin{tabular}{r|l}
            \textbf{Expediente} & \textbf{Nombre} \\ \hline
            219208106 & Bórquez Guerrero Angel Fernando \\
        \end{tabular}
    \end{center}

    \rule{\linewidth}{0.3mm}

    \begin{enumerate}
        % - Problema 1
        \item Traduzca las siguientes oraciones a fbf de LPO \par
             % Respuestas:
            \vspace{\Pspace} \par
            a) Cualquier que sea persistente puede aprender lógica
            { \color{azul}
                \begin{itemize}
                    \item $P(x) = $ ``$x$ es persistente"
                    \item $A(x) = $ ``$x$ puede aprender lógica"
                \end{itemize}
                \hspace*{2em} $\forall x (P(x) \rightarrow A(x))$
            }

            \vspace{\Pspace} \par
            b) Ningún político es honesto.
            { \color{azul} 
                \begin{itemize}
                    \item $P(x) = $ ``$x$ es político"
                    \item $H(x) = $ ``$x$ es honesto"
                \end{itemize}
                \hspace*{2em} $\lnot \exists x (P(x) \land H(x)) \equiv \forall x (P(x) \rightarrow \lnot H(x))$
            }

            \vspace{\Pspace} \par
            c) No todas las aves pueden volar.
            { \color{azul} 
                \begin{itemize}
                    \item $A(x) = $ ``$x$ es ave"
                    \item $V(x) = $ ``$x$ puede volar"
                \end{itemize}
                \hspace*{2em} $\lnot \forall x (A(x) \rightarrow V(x)) \equiv \exists x (A(x) \land \lnot V(x))$
            }

            \vspace{\Pspace} \par
            d) Ningún ave puede volar.
            { \color{azul} 
                \begin{itemize}
                    \item $A(x) = $ ``$x$ es ave"
                    \item $V(x) = $ ``$x$ puede volar"
                \end{itemize}
                \hspace*{2em} $\lnot \exists x (A(x) \rightarrow V(x)) \equiv \forall x (A(x) \land \lnot V(x))$
            }
            
            \newpage
            \vspace{\Pspace} \par
            e) $x$ es trascendental si y solo si es irracional.
            { \color{azul} 
                \begin{itemize}
                    \item $T(x) = $ ``$x$ es trascendental"
                    \item $I(x) = $ ``$x$ es irracional"
                \end{itemize}
                \hspace*{2em} $\exists x (T(x) \leftrightarrow I(x))$
            }

            \vspace{\Pspace} \par
            f) Si cualquiera puede resolver el problema, Ernesto puede.
            { \color{azul} 
                \begin{itemize}
                    \item $R(x) = $ ``$x$ puede resolver el problema"
                    \item $e = $ Ernesto
                \end{itemize}
                \hspace*{2em} $\forall x (R(x) \rightarrow R(e))$
            }

            \vspace{\Pspace} \par
            g) Nadie ama a un perdedor.
            { \color{azul} 
                \begin{itemize}
                    \item $A(x) = $ ``$x$ ama a un perdedor"
                \end{itemize}
                \hspace*{2em} $\lnot \exists x A(x)$
            }

            \vspace{\Pspace} \par
            h) Nadie en la clase de estadística es más inteligente que cualquiera en la clase de lógica. 
            \vspace{-0.7cm}
            { \color{azul} 
                \begin{itemize}
                    \item $E(x) = $ ``$x$ está en clase de estadística"
                    \item $I(x,y) = $ ``$x$ es más inteligente que $y$"
                    \item $L(x) = $ ``$x$ está en clase de lógica"
                \end{itemize}
                \hspace*{2em} $\lnot \exists x \forall y ((E(x) \land L(y)) \rightarrow I(x,y))$
            }

            \vspace{\Pspace} \par
            i) Todos aman a alguien y nadie ama a todo el mundo, o alguien ama a todo el mundo y alguien no ama a nadie. 
            \vspace{-0.7cm}
            \\ { \color{azul} 
                \begin{itemize}
                    \item $A(x,y) = $ ``$x$ ama a $y$"
                \end{itemize}
                \hspace*{2em} $[\forall x \exists y A(x,y) \land \lnot \forall x \forall y A(x,y)] \lor [ \exists x \forall y A(x,y) \land \exists y \lnot \forall x A(y,x) ]$
            }


            \vspace{\Pspace} \par
            j) Puedes engañar a algunas personas todo el tiempo, y puedes engañar a todas las personas alguna vez, pero no puedes engañar a todas las personas al mismo tiempo. 
            \vspace{-0.7cm}
            \\ { \color{azul} 
                \begin{itemize}
                    \item $E(x),y = $ ``Puedes engañar $x$ personas $y$ vez"
                \end{itemize}
                \hspace*{2em} $[ \exists x \forall y E(x,y) \land \forall x \exists y E(x,y) ] \lor [ \forall x \forall y \lnot E(x,y)]$
            }


            \vspace{\Pspace} \par
            k) Cualesquiera conjuntos que tengan los mismos elementos son iguales.
            \vspace{-0.7cm}
            \\ { \color{azul} 
                \begin{itemize}
                    \item $P(x,y) = $ ``$x$ pertenece a $y$"
                \end{itemize}
                \hspace*{2em} $\forall x \forall y (x = y \leftrightarrow \forall z(P(z,x) \leftrightarrow P(z,x)))$
            }

            \newpage
            \vspace{\Pspace} \par
            l) Toda persona que conoce a Julia la ama.
            { \color{azul} 
                \begin{itemize}
                    \item $C(x) = $ ``$x$ conoce a Julia"
                    \item $A(x) = $ ``$x$ ama a Julia"
                \end{itemize}
                \hspace*{2em} $\forall x (C(x) \rightarrow A(x))$
            }

            \vspace{\Pspace} \par
            m) No existe un conjunto que pertenezca exactamente a aquellos conjuntos que no pertenecen a sí mismos. 
            { \color{azul} 
                \begin{itemize}
                    \item $P(x) = $ ``$x$ es un conjunto que pertenece a sí mismo"
                \end{itemize}
                \hspace*{2em} $\lnot \exists x (P(x) \land \lnot P(x))$
            }

            \vspace{\Pspace} \par
            n) No existe un barbero que afeite precisamente a aquellos hombres que no se afeitan a sí mismos.
            { \color{azul} 
                \begin{itemize}
                    \item $A(x) = $ ``$x$ es un barbero que se afeita a sí mismo"
                \end{itemize}
                \hspace*{2em} $\lnot \exists x (A(x) \land \lnot A(x))$
            }

        % - Problema 
        \vspace{\Pspace}
        \item Traduzca a lenguaje natural las siguientes fbf. \par
             % Respuestas:
            \vspace{\Pspace} \par
            a) $ \forall x (M(x) \land \forall y \lnot W(x,y) \rightarrow U(x)) $
            \begin{itemize}
                \item $M(x) = $ ``$x$ es un hombre"
                \item $W(x,y) = $ ``$x$ está casado con $y$"
                \item $U(x) = $ ``$x$ es infeliz"
            \end{itemize}
            \par { \color{azul} Todas las personas que son hombres y no están casados con cualquier otra persona son infelices. }

            \vspace{\Pspace} \par
            b) $\forall x (V(x) \land P(x) \rightarrow A(x,b))$
            \begin{itemize}
                \item $V(x) = $ ``$x$ es un entero par"
                \item $P(x) = $ ``$x$ es un primo entero"
                \item $A(x,y) = $ ``\(x = y\)" y $b$ denota 2
            \end{itemize}
            \par { \color{azul} Todo número que es entero par y primo entero es igual a dos. }

            \vspace{\Pspace} \par
            c) $\lnot \exists y (I(y) \land \forall x (I(x) \rightarrow L(x,y)))$
            \begin{itemize}
                \item $I(y) = $ ``$y$ es un entero"
                \item $L(x, y) = x \leq y$ 
            \end{itemize}
            \par { \color{azul} No existe número entero que cualquier otro numero sea menor o igual. }

            \newpage
            d) En las siguientes fbf: 
            \begin{itemize}
                \item $A^{1}_{1}(x) = $ ``$x$ es una persona"
                \item $A^{2}_{1}(x, y) = $ ``$x$ odia a $y$"
            \end{itemize}
            \par \hspace*{2em} i. $\exists x (A^{1}_{1} x \land \forall y (A^{1}_{1} y \rightarrow A^{2}_{1}(x,y)))$
            \par \hspace*{2em} { \color{azul} ``Existe una persona que odia a todas las otras personas" }

            \par \hspace*{2em} ii. $\forall x (A^{1}_{1} x \rightarrow \forall y (A^{1}_{1} y \rightarrow A^{2}_{1}(x,y)))$
            \par \hspace*{2em} { \color{azul} ``Toda persona odia a todas las personas" }

            \par \hspace*{2em} iii. $\exists x (A^{1}_{1} x \land \forall y (A^{1}_{1} y \rightarrow (A^{2}_{1}(x,y) \leftrightarrow A^{2}_{1}(x,y))))$
            \par \hspace*{2em} { \color{azul} ``Existe una persona que odia a todas las demás solo si las odia" }



    \end{enumerate}
\end{document}
