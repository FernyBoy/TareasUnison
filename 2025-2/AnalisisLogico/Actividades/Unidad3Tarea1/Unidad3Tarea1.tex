\documentclass[a4paper, 11pt]{article}

% -- Coments
\usepackage{verbatim}

% ----- Fonts -----
% -- Fuente --
% \usepackage{fontspec}
% \setmonofont{JetBrainsMono Nerd Font}  

% -- Color --
\usepackage{xcolor}
%\definecolor{azul}{RGB}{00,33,99}
\definecolor{azul}{RGB}{35,72,180}

% -- Page Margin --
\usepackage[margin=1in]{geometry}

% -- Espaciados --
\newcommand{\Pspace}{0.5cm}
\newcommand{\Aspace}{0.2cm}

% -- Columnas --
\usepackage{multicol}

% -- Imagenes --
\usepackage{graphicx}
\usepackage{float}

% -- Matemáticas --
\usepackage{amsmath, amssymb}

% -- Gráficas --
\usepackage{pgfplots}
\pgfplotsset{compat=1.18}

% -- Código --
\usepackage{listings}
\lstset{
    language=C++,                   % Lenguaje del código
    basicstyle=\ttfamily\small,     % Fuente del código
    keywordstyle=\color{blue},      % Color de palabras clave
    commentstyle=\color{gray},      % Color de comentarios
    stringstyle=\color{red},        % Color de cadenas
    numbers=left,                   % Números de línea a la izquierda
    numberstyle=\tiny\color{gray},
    breaklines=true,                % Permitir saltos de línea
    frame=single                    % Marco alrededor del código
}

\begin{comment}
-- Formato de pregunta simple
    % - Problema n
    \vspace{\Pspace}
    \item Problema \par
        % Respuesta:
        \vspace{\Aspace} \par
        { \color{azul}  }


 -- Formato de pregunta multimple
        % - Problema 
        \vspace{\Pspace}
        \item  \par
             % Respuestas:
            \vspace{\Aspace} \par
            a) 
            \\ { \color{azul} }

            \vspace{\Aspace} \par
            b) 
            \\ { \color{azul} }


 -- Formato de imagen
\begin{figure}[!ht]
    \centering
    \includegraphics[width=0.5\textwidth]{}
\end{figure}


 -- Formato de tabla
\begin{tabular}{c|c}
    \textbf{A} & \textbf{B} \\
    \hline
    Texto & Texto \\
    Texto & Texto
\end{tabular} \\
\end{comment}


\title
{
    Análisis Lógico 2025-2 \\
    Tarea 1
}

\begin{document}

    \maketitle

    \begin{center}
        \begin{tabular}{r|l}
            \textbf{Expediente} & \textbf{Nombre} \\ \hline
            219208106 & Bórquez Guerrero Angel Fernando \\
        \end{tabular}
    \end{center}

    \rule{\linewidth}{0.3mm}

    \vspace{0.3cm}

    \begin{centering}
    Transformar a cláusulas (FNCS) las siguientes fórmulas e identificar qué tipo de cláusulas son.
    \end{centering}

    \begin{enumerate}
    % - Problema 1
    \vspace{\Pspace}
    \item $\forall x [P(x) \rightarrow P(x)]$ \par
        % Respuesta:
        \vspace{\Aspace} \par
        { \color{azul} 
            $\forall x [\lnot P(x) \lor P(x) ]$
            \par $C_{1} = \{ \lnot P(x), P(x) \}$: Cláusula mixta y Horn
        }



    % - Problema 2
    \vspace{\Pspace}
    \item $[\lnot[\forall x P(x)]] \rightarrow \exists x P(X)$ \par
        % Respuesta:
        \vspace{\Aspace} \par
        { \color{azul} 
            $\lnot [ \lnot (\forall x P(x))] \lor \exists x P(x)$
            \par $\forall x P(x) \lor \exists x P(x)$
            \par $\forall x P(x) \lor \exists y P(y)$
            \par $\forall x \exists y (P(x) \lor P(y))$
            \par $\forall x (P(x) \lor P(f(x))$
            \par $C_{1} = \{ P(x), P(f(x)) \}$: Cláusula positiva y no Horn
        }

    

    % - Problema 3
    \vspace{\Pspace}
    \item $\lnot \forall x [P(x) \rightarrow [\forall y (P(y) \rightarrow P(f(x,y)) \land (\lnot \forall y (Q(x,y) \rightarrow P(y))]]$ \par
        % Respuesta:
        \vspace{\Aspace} \par
        { \color{azul} 
            \par $\forall y [\lnot \forall x [P(x) \rightarrow [\forall z (P(z) \rightarrow P(f(x,z))) \land (\lnot \forall w Q(x,w) \rightarrow P(y))]]]$
            \par $\forall y \exists x \lnot [\lnot P(x) \lor [\forall z (\lnot P(z) \lor P(f(x,z))) \land (\lnot \lnot \forall w Q(x,w) \lor P(y))]]$
            \par $\forall y \exists x [P(x) \land \lnot [\forall z (\lnot P(z) \lor P(f(x,z))) \land (\forall w Q(x,w) \lor P(y)]]$
            \par $\forall y \exists x [P(x) \land [\exists z \lnot (\lnot P(z) \lor P(f(x,z))) \lor \lnot (\forall w Q(x,w) \lor P(y)]]$
            \par $\forall y \exists x [P(x) \land [\exists z (P(z) \land \lnot P(f(x,z))) \lor (\exists w \lnot Q(x,w) \land \lnot P(y)]]$
            \par $\forall y \exists x [P(x) \land [\exists z (P(z) \land \lnot P(f(x,z))) \lor \exists w (\lnot Q(x,w) \land \lnot P(y)]]$
            \par $\forall y \exists x [P(x) \land \exists z \exists w [(P(z) \land \lnot P(f(x,z))) \lor (\lnot Q(x,w) \land \lnot P(y)]]$
            \par $\forall y \exists x \exists z \exists w [P(x) \land [(P(z) \land \lnot P(f(x,z))) \lor (\lnot Q(x,w) \land \lnot P(y)]]$
            \par $\forall y [P(g(y)) \land [(P(h(y)) \land \lnot P(f(g(y),h(y)))) \lor (\lnot Q(g(y),i(y)) \land \lnot P(y)]]$
            \par $\forall y [P(g(y) \land ( P(g(y)) \lor \lnot Q(g(y),i(y)) ) \land ( P(h(y)) \lor \lnot P(y) ) \land ( \lnot P(f(g(y),h(y))) \lor \lnot Q(g(y),i(y)) ) \land ( \lnot P(f(g(y),h(y))) \lor \lnot P(y) )]$
            \par $C_{1} = \{ P(g(y) \}$: Cláusula unitaria, positiva y Horn
            \par $C_{2} = \{ P(g(y)), \lnot Q(g(y),i(y)) \}$: Cláusula mixta y Horn
            \par $C_{3} = \{ P(h(y)), \lnot P(y) \}$: Cláusula mixta y Horn
            \par $C_{4} = \{ \lnot P(f(g(y),h(y))), \lnot Q(g(y),i(y)) \}$: Cláusula negativa y no Horn
            \par $C_{5} = \{ \lnot P(f(g(y),h(y))), \lnot P(y) \}$: Cláusula negativa y no Horn
        }



    % - Problema 4
    \vspace{\Pspace}
    \item $\exists x [\forall y \exists z P(x,y,z) \land \exists z \forall y \lnot P(x,y,z)]$ \par
        % Respuesta:
        \vspace{\Aspace} \par
        { \color{azul}  
            $\exists x [\forall y \exists z P(x,y,z) \land \exists u \forall w \lnot P(x,w,u)]$
            \par $\exists x \forall y \exists z \exists u \forall w [P(x,y,z) \land \lnot P(x,w,u)]$
            \par $\forall y \exists z \exists u \forall w [P(c_{1},y,z) \land \lnot P(c_{1},w,u)]$
            \par $\forall y \exists u \forall w [P(c_{1},y,f(y)) \land \lnot P(c_{1},w,u)]$
            \par $\forall y \forall w [P(c_{1},y,f(y)) \land \lnot P(c_{1},w,g(y))]$
            \par $C_{1} = \{ P(c_{1},y,f(y)) \}$: Cláusula unitaria, positiva y Horn
            \par $C_{2} = \{ \lnot P(c_{1},w,g(y)) \}$: Cláusula unitaria, negativa y no Horn
        }



    % - Problema 5
    \vspace{\Pspace}
    \item $\forall x [\exists y Q(x,y) \lor \forall y \exists z P(x,y,z)]$ \par
        % Respuesta:
        \vspace{\Aspace} \par
        { \color{azul}  
            $\forall x [ \exists y Q(x,y) \lor \forall u \exists z P(x,u,z) ]$
            \par $\forall x [ \exists y \forall u \exists z ( Q(x,y) \lor P(x,u,z) ) ]$
            \par $\forall x \exists y \forall u \exists z ( Q(x,y) \lor P(x,u,z) )$
            \par $\forall x \forall u \exists z ( Q(x, f(x)) \lor P(x,u,z) )$
            \par $\forall x \forall u ( Q(x,f(x)) \lor P(x,u,g(x,u)) )$
            \par $C_{1} = \{ Q(x,f(x)), P(x,u,g(x,u)) \}$: Cláusula positiva y no Horn
        }



    % - Problema 6
    \vspace{\Pspace}
\item $\lnot [[(P(x) \rightarrow Q(x)) \rightarrow \lnot Q(x)] \rightarrow \lnot Q(x)]$ \par
        % Respuesta:
        \vspace{\Aspace} \par
        { \color{azul}  
            $\lnot [ \lnot [ \lnot ( \lnot P(x) \lor Q(x)) \lor \lnot Q(x)] \lor \lnot Q(x)]$
            \par $[(P(x) \land \lnot Q(x)) \lor \lnot Q(x)] \land Q(x)$
            \par $[(P(x) \lor \lnot Q(x)) \land (\lnot Q(x) \lor \lnot Q(x))] \land Q(x)$
            \par $(P(x) \lor \lnot Q(x)) \land \lnot Q(x) \land Q(x)$
            \par $C_{1} = \{ P(x), \lnot Q(x) \}$: Cláusula mixta y Horn
            \par $C_{2} = \{ \lnot Q(x) \}$: Cláusula unitaria, negativa y no Horn
            \par $C_{3} = \{ Q(x) \}$: Cláusula unitaria, positiva y Horn
        }
    \end{enumerate}
\end{document}
