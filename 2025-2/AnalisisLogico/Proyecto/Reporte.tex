\documentclass[a4paper, 11pt]{article}

% -- Coments
\usepackage{verbatim}

% ----- Fonts -----
% -- Fuente --
% \usepackage{fontspec}
% \setmonofont{JetBrainsMono Nerd Font}  

% -- Color --
\usepackage{xcolor}
%\definecolor{azul}{RGB}{00,33,99}
\definecolor{azul}{RGB}{35,72,180}

% -- Page Margin --
\usepackage[margin=1in]{geometry}

% -- Espaciados --
\newcommand{\Pspace}{0.5cm}
\newcommand{\Aspace}{0.2cm}

% -- Columnas --
\usepackage{multicol}

% -- Imagenes --
\usepackage{graphicx}
\usepackage{float}

% -- Matemáticas --
\usepackage{amsmath, amssymb}

% -- Gráficas --
\usepackage{pgfplots}
\pgfplotsset{compat=1.18}

% -- Código --
\usepackage{listings}
\lstset{
    language=C++,                   % Lenguaje del código
    basicstyle=\ttfamily\small,     % Fuente del código
    keywordstyle=\color{blue},      % Color de palabras clave
    commentstyle=\color{gray},      % Color de comentarios
    stringstyle=\color{red},        % Color de cadenas
    numbers=left,                   % Números de línea a la izquierda
    numberstyle=\tiny\color{gray},
    breaklines=true,                % Permitir saltos de línea
    frame=single                    % Marco alrededor del código
}

\title
{
    Análisis Lógico 2025-2 \\
    Árbol genealógico
}

    \begin{document}

    \maketitle

    \begin{center}
        \begin{tabular}{r|l}
            \textbf{Expediente} & \textbf{Nombre} \\ \hline
            219208106 & Bórquez Guerrero Angel Fernando \\
            223215039 & Miranda Sánchez Javier Leonardo

        \end{tabular}
    \end{center}

    \rule{\linewidth}{0.3mm}

    \vspace{0.3cm}

    \section{Descripción del programa}
    El programa desarrollado en Prolog construye un árbol genealógico de tres generaciones mediante hechos que representan las relaciones familiares básicas: género, paternidad y maternidad. A partir de estos hechos, se definieron reglas lógicas que permiten inferir relaciones familiares más complejas, como progenitor, abuelo, hermano, tío y primo.

    Estas reglas se basan en la capacidad de Prolog para realizar inferencias por medio del encadenamiento lógico. Con ellas es posible realizar consultas que determinen parentescos dentro del árbol familiar, permitiendo responder automáticamente preguntas relacionadas con descendencia, parentesco entre hermanos, relaciones de tíos y sobrinos, así como vínculos entre primos.


\section*{Consultas y Explicaciones}

A continuación se presentan ocho consultas realizadas al programa en Prolog, junto con su respectiva explicación en lenguaje natural.

\subsection*{1. ¿Quiénes son los abuelos de Andrés?}

\begin{verbatim}
?- abuelo(X, andres).
\end{verbatim}

Esta consulta obtiene todos los hombres que son abuelos de Andrés. La regla identifica a los padres del padre o la madre de Andrés, filtrando únicamente a los abuelos varones.

\subsection*{2. ¿Quiénes son las abuelas de Melina?}

\begin{verbatim}
?- abuela(X, melina).
\end{verbatim}

Esta consulta devuelve a todas las mujeres que son abuelas de Melina. La regla busca mujeres que sean progenitoras de alguno de los padres de Melina.

\subsection*{3. ¿Quiénes son los hermanos de Karol?}

\begin{verbatim}
?- hermanos(X, karol).
\end{verbatim}

Con esta consulta se obtienen todos los hermanos y hermanas de Karol, ya que la regla \texttt{hermanos/2} combina tanto hermanos hombres como hermanas mujeres.

\subsection*{4. ¿Qué tías tiene Alan?}

\begin{verbatim}
?- tia(X, alan).
\end{verbatim}

Esta consulta encuentra a todas las mujeres que son hermanas de alguno de los padres de Alan. Cualquier mujer que cumpla esta condición es considerada su tía.

\subsection*{5. ¿Qué tíos tiene Víctor?}

\begin{verbatim}
?- tio(X, victor).
\end{verbatim}

La consulta obtiene a todos los hombres que son hermanos del padre o madre de Víctor. La regla filtra exclusivamente a los tíos varones.

\subsection*{6. ¿Quiénes son los primos de Michelle?}

\begin{verbatim}
?- primo(X, michelle).
\end{verbatim}

Aquí se obtienen todos los hombres que son hijos de los hermanos de los padres de Michelle. La regla solo devuelve primos varones.

\subsection*{7. ¿Quiénes son las primas de Alejandro?}

\begin{verbatim}
?- prima(X, alejandro).
\end{verbatim}

Esta consulta devuelve a todas las mujeres que son hijas de los hermanos de los padres de Alejandro. La regla filtra únicamente primas mujeres.

\subsection*{8. ¿Existe una relación de hermandad entre Emi y Michelle?}

\begin{verbatim}
?- hermanos(emi, michelle).
\end{verbatim}

Esta consulta verifica si Emi y Michelle comparten al menos un progenitor. Si Prolog responde \texttt{true}, significa que son hermanas; si responde \texttt{false}, no existe relación de hermandad entre ellas.

\end{document}
