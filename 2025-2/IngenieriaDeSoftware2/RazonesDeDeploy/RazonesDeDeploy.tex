\documentclass[12pt,a4paper]{article}
\usepackage[utf8]{inputenc}
\usepackage[T1]{fontenc}
\usepackage[spanish]{babel}
\usepackage{setspace}
\usepackage{geometry}
\geometry{margin=1.5cm}
\setlength{\parskip}{1em}
\setlength{\parindent}{0em}

\title{Resumen: Razones para Escoger el Stage de Deploy en DevOps 2.0}
\author{Equipo de Ingeniería de Software 2}
\date{\today}

\begin{document}

\maketitle

En nuestro equipo decidimos enfocarnos en el \textbf{stage de Deploy} dentro de DevOps 2.0 debido a que representa el punto en el que los esfuerzos de todo el ciclo de desarrollo se materializan en un producto disponible para los usuarios finales. Consideramos que el despliegue es un momento crítico, ya que de su correcta ejecución depende la estabilidad, la calidad y la confiabilidad del software entregado.  

Uno de los aspectos que más valoramos es la \textbf{automatización} que caracteriza a esta fase. Gracias a prácticas como la Integración y el Despliegue Continuo (CI/CD), es posible reducir los errores humanos, acelerar la entrega de nuevas funcionalidades y responder con mayor rapidez a los cambios del mercado (IEEE, 2017). Esto no solo mejora la eficiencia operativa, sino que también aumenta la competitividad de las organizaciones.  

Asimismo, observamos que el Deploy permite implementar \textbf{estrategias avanzadas}, como despliegues graduales o pruebas en entornos controlados, lo que facilita minimizar riesgos y garantizar una experiencia más estable para los usuarios (Chatterjee \& Mittal, 2024). Además, esta fase no se limita a lo técnico: implica también cambios culturales y organizacionales, fomentando la colaboración entre equipos de desarrollo y operaciones (Febrianto, Nugraheni, Suharto, Widodo \& Ariyanti, 2024).  

En conclusión, elegimos el stage de Deploy porque lo vemos como el puente decisivo entre la innovación y el usuario final, donde la automatización, la seguridad y la eficiencia se integran para ofrecer software de calidad y con un impacto directo en la satisfacción del cliente.

\section*{Referencias}

\begin{itemize}
    \item \label{systematic2017} IEEE. (2017). \textit{Continuous Integration, Delivery and Deployment: A Systematic Review on Approaches, Tools, Challenges and Practices}. IEEE Xplore. \\ Disponible en: \url{https://ieeexplore.ieee.org/document/7884954}
    \item \label{chatterjee2024} Chatterjee, P. S., \& Mittal, H. K. (2024). \textit{Enhancing operational efficiency through the integration of CI/CD and DevOps in software deployment}. \\ IEEE. DOI: \url{https://doi.org/10.1109/CCICT62777.2024.00038}
    \item \label{febrianto2024} Febrianto, R. F., Nugraheni, D. M. K., Suharto, E., Widodo, A. P., \& Ariyanti, Y. D. P. (2024). \textit{Deployment strategy with integration testing implementation using DevOps method in development and production environment}. 7th International Conference on Informatics and Computational Sciences (ICICoS). \\ IEEE. DOI: \url{https://doi.org/10.1109/ICICoS62600.2024.10636830}
\end{itemize}

\end{document}

