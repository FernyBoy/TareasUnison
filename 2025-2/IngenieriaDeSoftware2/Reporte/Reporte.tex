\documentclass[12pt,a4paper]{article}
\usepackage{setspace}
\usepackage{geometry}
\geometry{margin=1.5cm}
\setlength{\parskip}{1em}
\setlength{\parindent}{0em}

\usepackage[style=apa,sorting=nyt]{biblatex}
\addbibresource{referencias.bib}

\author{Equipo de Ingeniería de Software 2}
\date{\today}

\begin{document}
\begin{center}
    \begin{tabular}{r|l}
        \textbf{Expediente} & \textbf{Nombre} \\ \hline
        220210296 & Altamirano Ocejo Michell Berenice \\
        219208106 & Bórquez Guerrero Angel Fernando \\
        223216515 & Giron Leon Gibran \\
        223207958 & Molina Buzame Danna Valeria \\
        223204115 & Ruiz Beltran Oliver
    \end{tabular}
\end{center}

¿Qué es DevOps? \\
Es un “conjunto de principios y prácticas que permiten una mejor comunicación y colaboración entre las partes interesadas relevantes con el fin de especificar, desarrollar y operar productos y servicios de software y sistemas, y mejoras continuas en todos los aspectos del ciclo de vida”. 
\parencite{ieee2021standard}.

¿Por qué es importante? \\
Permite una mayor eficiencia en la entrega de software al automatizar los procesos, reduciendo errores y acelerando los tiempos. Mejora la calidad del software gracias a la integración y entrega continua, que facilitan la detección y corrección temprana de errores. Fomenta la colaboración entre los equipos de desarrollo y operaciones, mejorando la comunicación y la toma de decisiones. Aumenta la capacidad de respuesta ante cambios en los requisitos del negocio y del mercado, y contribuye a una mejor experiencia del cliente al entregar software de forma más rápida y confiable. 
\parencite{nivelics2023}

¿Qué etapa seleccionaron? \\
Elegimos el stage de Deploy porque lo vemos como el puente decisivo entre la innovación y el usuario final, donde la automatización, la seguridad y la eficiencia se integran para ofrecer software de calidad y con un impacto directo en la satisfacción del cliente.

¿Qué problemas identificaron o qué conocimientos y habilidades se requieren? \\
En un entorno de DevOps, el deployment requiere habilidades en automatización, como es usando herramientas CI/CD, control de versiones como Git, manejo de contenedores, conocimientos sólidos sobre el área del desarrollo, prueba y producción.
Los problemas más comunes incluyen fallos por configuración incorrecta, errores  no detectados en código, despliegues lo que puede afectar la estabilidad y continuidad del servicio. 

¿Qué objetivo tiene su trabajo? \\
El objetivo de identificar los conocimientos y habilidades requeridas para el deployment en nuestra investigación es demostrar que un despliegue confiable y eficiente depende directamente de competencias técnicas específicas. Estas habilidades permiten automatizar procesos, garantizar la seguridad, mejorar la calidad del software y facilitar la integración continua en entornos modernos. Además, ayudan a vincular la teoría del deployment con las necesidades prácticas de equipos DevOps, lo que refuerza la relevancia de su estudio y aplicación en proyectos reales.

¿Cómo contribuye cada artículo seleccionado a lograr el objetivo? \\
En conjunto, los artículos seleccionados aportan diferentes perspectivas y soluciones que ayudan a entender cómo DevOps, la integración continua (CI), el despliegue continuo (CD) y la automatización pueden aplicarse de manera más efectiva en la industria del software.
Algunos trabajos se centran en los retos culturales y organizacionales que existen al adoptar DevOps, destacando la importancia de la colaboración entre equipos y la necesidad de modelos de madurez que guíen la implementación. Otros abordan aspectos técnicos, como las herramientas de CI/CD, la automatización de pipelines y las estrategias de despliegue seguro y confiable. También hay artículos que muestran cómo DevOps se aplica en áreas específicas, como el Internet de las Cosas (IoT), los sistemas ciber-físicos o entornos en la nube, lo cual amplía la visión hacia distintos contextos industriales.

\newpage
Altamirano Ocejo Michell Berenice \\
\rule{\linewidth}{0.3mm} \\
• Artículo 1 \\
AIDOaRt: AI-augmented Automation for DevOps, a Model-based Framework for Continuous Development in Cyber-Physical Systems. (2021, 1 septiembre). IEEE Conference Publication | IEEE Xplore. https://ieeexplore.ieee.org/document/9556443

• Artículo 2 \\
Challenges of Adopting DevOps Culture on the Internet of Things Applications - A Solution Model. (2022, 10 octubre). IEEE Conference Publication | IEEE Xplore. https://ieeexplore.ieee.org/document/9988182


Elegí estos dos artículos porque se complementan muy bien. El primero trata sobre DevOps en IoT, que es un tema muy actual ya que cada vez usamos más dispositivos conectados. Este artículo propone un modelo de madurez (IDMM) que funciona como una guía para evaluar qué tan bien se está aplicando DevOps y cómo mejorar paso a paso.

El segundo aborda cómo la inteligencia artificial puede potenciar DevOps en sistemas ciber-físicos, como autos, trenes o sistemas industriales. Ahí presentan un framework (AIDOaRt) que usa IA para automatizar procesos dentro de la pipeline, como pruebas, monitoreo y despliegues.

Al ponerlos juntos, tengo una visión doble: uno más práctico y enfocado en el presente (IoT) y otro más innovador que muestra hacia dónde puede evolucionar DevOps con IA. Los dos cumplen con lo que pide el tema, porque hablan de CI/CD, automatización, deployment y modelos de referencia.



\vspace{1.5cm}
Bórquez Guerrero Angel Fernando \\
\rule{\linewidth}{0.3mm} \\
• Artículo 1 \\
DevOps Adoption in Software Development Organizations: A Systematic Literature Review. (2024, 21 febrero). IEEE Conference Publication | IEEE Xplore. https://ieeexplore.ieee.org/document/10499789

• Artículo 2 \\
The DevOps Reference Architecture Evaluation : A Design Science Research Case Study. (2020, 1 agosto). IEEE Conference Publication | IEEE Xplore. https://ieeexplore.ieee.org/document/9192390

DevOps Adoption in Software Development Organizations ofrece una revisión sistemática de la literatura sobre la adopción de DevOps en organizaciones de desarrollo de software. Analiza diversos estudios y casos para identificar patrones, desafíos y mejores prácticas en la implementación de DevOps. Proporciona una visión amplia y basada en evidencia sobre cómo las organizaciones están adoptando DevOps, incluyendo el stage de despliegue y la integración de CI/CD.

The DevOps Reference Architecture Evaluation presenta una evaluación de una arquitectura de referencia para DevOps mediante un estudio de investigación en diseño. Proporciona una perspectiva estructurada sobre cómo las prácticas de DevOps pueden ser implementadas y evaluadas en un entorno real. Su enfoque en la arquitectura de referencia es esencial para comprender cómo se organiza y estructura el proceso de despliegue en DevOps, lo que es directamente aplicable a la automatización y CI/CD.

\newpage
Giron Leon Gibran \\
\rule{\linewidth}{0.3mm} \\
• Artículo 1 \\
32675-2021 - IEEE Standard for DevOps:Building Reliable and Secure Systems Including Application Build, Package, and Deployment. (2021, 16 abril). IEEE Standard | IEEE Xplore. https://ieeexplore.ieee.org/document/9415476

• Artículo 2 \\
Standards-Based DevOps Automation and Integration Using TOSCA. (2014, 1 diciembre). IEEE Conference Publication | IEEE Xplore. https://ieeexplore.ieee.org/document/7027481


Estos artículos  ofrecen enfoques complementarios y bien fundamentados. El primero, “Standards-Based DevOps Automation and Integration Using TOSCA”, presenta una propuesta técnica y práctica basada en el estándar TOSCA, que permite automatizar y orquestar despliegues en entornos cloud de manera estandarizada. Su enfoque en la integración de herramientas heterogéneas y el uso de casos prácticos lo hacen ideal para comprender cómo aplicar DevOps en la práctica. Por otro lado, el estándar “IEEE 32675-2021” proporciona una guía formal y estructurada para implementar DevOps de forma segura y confiable. Abarca todo el ciclo de vida del software, desde la construcción hasta el despliegue, y se basa en normas internacionales, lo que garantiza su aplicabilidad en entornos reales. En conjunto, ambos documentos ofrecen una visión técnica y normativa clave para abordar el deployment en DevOps.



\vspace{1.5cm}
Danna Valeria Molina Buzame \\
\rule{\linewidth}{0.3mm} \\
• Artículo 1 \\
Continuous Deployment in IoT Edge Computing : A GitOps implementation. (2022, 22 junio). IEEE Conference Publication | IEEE Xplore. https://ieeexplore.ieee.org/document/9820108

• Artículo 2 \\
Efficient Application Deployment: GitOps for Faster and Secure CI/CD Cycles. (2024, 16 mayo). IEEE Conference Publication | IEEE Xplore. https://ieeexplore.ieee.org/document/10582118


Escogí estos dos artículos porque juntos me ayudan a darle una base a la investigación sobre deployment.

 El primero habla de cómo GitOps mejora los procesos de integración y despliegue continuo (CI/CD), haciéndolos más rápidos, seguros y confiables (S. R. J., D. P. P. S., A. M. J., C. A., \& Shanthamalar, 2024). Me parece muy útil porque explica la parte teórica y metodológica, mostrando cómo Git puede ser la fuente principal de control y trazabilidad en los despliegues. Esto conecta directamente con el objetivo de la investigación, que busca proponer un deployment más ordenado y seguro.
El segundo artículo no se queda solo en la teoría, sino que presenta un caso aplicado en algunos entornos (López-Viana, Díaz, \& Pérez, 2022). Ahí muestran cómo GitOps puede resolver problemas reales de deployment en sistemas distribuidos, que suelen ser más complejos por temas como la latencia o la gran cantidad de nodos. Esto sirve mucho porque da un ejemplo práctico de cómo aplicar lo que plantea el primer artículo en escenarios más cercanos a la realidad.


\newpage
Ruiz Beltran Oliver \\
\rule{\linewidth}{0.3mm} \\
• Artículo 1 \\
Overcoming Challenges With Continuous Integration and Deployment Pipelines: An Experience Report From a Small Company. (2020, 1 junio). IEEE Journals \& Magazine | IEEE Xplore. https://ieeexplore.ieee.org/document/8866741


• Artículo 2 \\
DevSecOps in Finance: Strengthening the Security Model of Applications. (2024, 22 marzo). IEEE Conference Publication | IEEE Xplore. https://ieeexplore.ieee.org/document/10502917


El primero, “Overcoming Challenges With Continuous Integration and Deployment Pipelines: An Experience Report From a Small Company”, describe de manera práctica los problemas de migrar problemas, destacando los retos en la automatización, la integración continua y los pipelines de despliegue. Por otro lado, “DevSecOps in Finance: Strengthening the Security Model of Applications” se enfoca en la seguridad dentro del ciclo de vida del software y la relevancia de incorporar controles de seguridad desde las primeras fases del desarrollo, alineando prácticas de DevOps con las necesidades de sectores altamente regulados como el financiero.
Estos dos artículos en conjunto ofrecen una visión amplia y complementaria sobre la aplicación de DevOps en escenarios reales.

\vspace{1.5cm}
\printbibliography

\end{document}
