\documentclass[a4paper, 11pt]{article}

% -- Language --
\usepackage[spanish]{babel}
\usepackage[utf8]{inputenc}

% ----- Fonts -----
% -- Fuente --
% \usepackage{fontspec}
% \setmonofont{JetBrainsMono Nerd Font}  

% -- Color --
\usepackage{xcolor}
%\definecolor{azul}{RGB}{00,33,99}
\definecolor{azul}{RGB}{35,72,180}

% -- Page Margin --
\usepackage[margin=1in]{geometry}

% -- Espaciados --
\newcommand{\Pspace}{0.5cm}
\newcommand{\Aspace}{0.2cm}

% -- Columnas --
\usepackage{multicol}

% -- Imagenes --
\usepackage{graphicx}
\usepackage{float}

% -- Matemáticas --
\usepackage{amsmath, amssymb}

% -- Gráficas --
\usepackage{pgfplots}
\pgfplotsset{compat=1.18}

% -- Código --
\usepackage{listings}
\lstset{
    language=C++,                   % Lenguaje del código
    basicstyle=\ttfamily\small,     % Fuente del código
    keywordstyle=\color{blue},      % Color de palabras clave
    commentstyle=\color{gray},      % Color de comentarios
    stringstyle=\color{red},        % Color de cadenas
    numbers=left,                   % Números de línea a la izquierda
    numberstyle=\tiny\color{gray},
    breaklines=true,                % Permitir saltos de línea
    frame=single                    % Marco alrededor del código
}



\title
{
    Diseño de Sistemas Digitales 2025-2 \\
    Problemas Unidad 4
}

    \begin{document}

    \maketitle

    \begin{center}
        \begin{tabular}{r|l}
            \textbf{Expediente} & \textbf{Nombre} \\ \hline
            219208106 & Bórquez Guerrero Angel Fernando \\
        \end{tabular}
    \end{center}

    \rule{\linewidth}{0.3mm}



    \vspace{0.3cm}
    \begin{enumerate}
        % - Problema 1
        \item Sume lo siguiente en binario. \par
             % Respuestas:
            \vspace{\Aspace} \par
            a) $0{.}1011 + 0{.}1111$
            \\ { \color{azul} }

            \vspace{\Aspace} \par
            b) $10011011 + 10011101$
            \\ { \color{azul} }

            \vspace{\Aspace} \par
            c) $1010.01 + 10.111$
            \\ { \color{azul} }



        % - Problema 2
        \vspace{\Pspace}
        \item Represente cada uno de los siguientes números decimales con signo en el sistema de complemento a 2. Use un total de ocho bits, incluyendo el bit de signo. \par
            % Respuestas:
            \vspace{\Aspace} \par
            a) $+1$
            \\ { \color{azul} }

            \vspace{\Aspace} \par
            b) $-128$
            \\ { \color{azul} }

            \vspace{\Aspace} \par
            c) $+169$
            \\ { \color{azul} }

            \vspace{\Aspace} \par
            d) $0$
            \\ { \color{azul} }

            \vspace{\Aspace} \par
            e) $+84$
            \\ { \color{azul} }

            \vspace{\Aspace} \par
            f) $-190$
            \\ { \color{azul} }
            
            \vspace{\Aspace} \par
            g) $+3$
            \\ { \color{azul} }

            \vspace{\Aspace} \par
            h) $-3$
            \\ { \color{azul} }



        % - Problema 3
        \item Cada uno de los siguientes números representa un número decimal con signo en el sistema de complemento a 2. Determine el valor decimal de los siguientes valores:
            % Respuestas:
            \vspace{\Aspace} \par
            a) $10000000$
            \\ { \color{azul} }

            \vspace{\Aspace} \par
            b) $11111111$
            \\ { \color{azul} }

            \vspace{\Aspace} \par
            c) $10000001$
            \\ { \color{azul} }

            \vspace{\Aspace} \par
            d) $01100011$
            \\ { \color{azul} }

            \vspace{\Aspace} \par
            e) $11011001$
            \\ { \color{azul} }



        % - Problema 4
        \vspace{\Pspace}
        \item La razón por la que el método de signo-magnitud para representar números con signo $n$ se utiliza en la mayoría de las computadoras puede ilustrarse mediante lo siguiente:.
            % Respuestas:
            \vspace{\Aspace} \par
            a) Represente +12 en ocho bits, utilizando la forma signo-magnitud.
            \\ { \color{azul} }

            \vspace{\Aspace} \par
            b) Represente -12 en ocho bits, utilizando la forma signo-magnitud.
            \\ { \color{azul} }

            \vspace{\Aspace} \par
            c) Sume los dos números binarios y observe que la suma no se parece en nada a cero.
            \\ { \color{azul} }


        \vspace{\Pspace}
        % - Problema 5
        \item Multiplique los siguientes pared de números binarios. \par
            % Respuestas:
            \vspace{\Aspace} \par
            a) $111 \times 101$
            \\ { \color{azul} }

            \vspace{\Aspace} \par
            b) $1011 \times 1011$
            \\ { \color{azul} }

            \vspace{\Aspace} \par
            c) $101{.}101 \times 110{.}010$
            \\ { \color{azul} }

            \vspace{\Aspace} \par
            d) $0.1101 \times 0.1011$
            \\ { \color{azul} }


    
        \newpage
        % - Problema 6
        \vspace{\Pspace}
        \item Realice las siguientes divisiones.
            % Respuesta:
            \vspace{\Aspace} \par
            a) $1100 \div 100$
            \\ { \color{azul} }

            \vspace{\Aspace} \par
            b) $111111 \div 1001$
            \\ { \color{azul} }

            \vspace{\Aspace} \par
            c) $10111 \div 100$
            \\ { \color{azul} }

            \vspace{\Aspace} \par
            d) $10110{.}1101 \div 1{.}1$
            \\ { \color{azul} }
            

        
        \vspace{\Pspace}
        % - Problema 7
        \item Escriba la tabla de funciones para un medio sumador, $HA$ (entradas $A$ y $B$; salidas SUMA y ACARREO). A partir de la tabla de funciones, diseñe un circuito lógico que actúe como medio sumador.
        \begin{figure}[!ht]
            \centering
            \includegraphics[width=0.5\textwidth]{Circuitos/Figuras/Figura_8.pdf}
        \end{figure}
            % Respuesta:
            \vspace{\Aspace} \par
            { \color{azul} }



        \newpage
        % - Problema 8
        \vspace{\Pspace}
        \item El desbordamiento ocurre cuando los dos números que se van a sumar o a restar producen un resultado que contiene más bits que la capacidad del acumulador. Diseñe un circuito lógico para el sumador de la figura que produzca una salida de 1 cada vez que ocurra la condición de desbordamiento.
        \begin{figure}[!ht]
            \centering
            \includegraphics[width=0.8\textwidth]{Circuitos/Figuras/Figura_9.pdf}
        \end{figure}
            % Respuesta:
            \vspace{\Aspace} \par
            { \color{azul} }
    \end{enumerate}
\end{document}
