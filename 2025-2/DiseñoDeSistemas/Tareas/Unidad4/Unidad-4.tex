\documentclass[a4paper, 11pt]{article}

% -- Language --
\usepackage[spanish]{babel}
\usepackage[utf8]{inputenc}

% ----- Fonts -----
% -- Fuente --
% \usepackage{fontspec}
% \setmonofont{JetBrainsMono Nerd Font}  

% -- Color --
\usepackage{xcolor}
%\definecolor{azul}{RGB}{00,33,99}
\definecolor{azul}{RGB}{35,72,180}

% -- Page Margin --
\usepackage[margin=1in]{geometry}

% -- Espaciados --
\newcommand{\Pspace}{0.5cm}
\newcommand{\Aspace}{0.2cm}

% -- Columnas --
\usepackage{multicol}

% -- Imagenes --
\usepackage{graphicx}
\usepackage{float}

% -- Matemáticas --
\usepackage{amsmath, amssymb}

% -- Gráficas --
\usepackage{pgfplots}
\pgfplotsset{compat=1.18}

% -- Código --
\usepackage{listings}
\lstset{
    language=C++,                   % Lenguaje del código
    basicstyle=\ttfamily\small,     % Fuente del código
    keywordstyle=\color{blue},      % Color de palabras clave
    commentstyle=\color{gray},      % Color de comentarios
    stringstyle=\color{red},        % Color de cadenas
    numbers=left,                   % Números de línea a la izquierda
    numberstyle=\tiny\color{gray},
    breaklines=true,                % Permitir saltos de línea
    frame=single                    % Marco alrededor del código
}



\title
{
    Diseño de Sistemas Digitales 2025-2 \\
    Problemas Unidad 4
}

    \begin{document}

    \maketitle

    \begin{center}
        \begin{tabular}{r|l}
            \textbf{Expediente} & \textbf{Nombre} \\ \hline
            219208106 & Bórquez Guerrero Angel Fernando \\
        \end{tabular}
    \end{center}

    \rule{\linewidth}{0.3mm}



    \vspace{0.3cm}
    \begin{enumerate}
        % - Problema 1
        \item Sume lo siguiente en binario. \par
             % Respuestas:
            \vspace{\Aspace} \par
            a) $0{.}1011 + 0{.}1111$
            \\ { \color{azul} 
                \begin{tabular}{r|}
                    $0{.}1011$ \\
                    $0{.}1111$ \\
                    \hline 
                    $1{.}1010$
                \end{tabular}
            }

            \vspace{\Aspace} \par
            b) $10011011 + 10011101$
            \\ { \color{azul} 
                \begin{tabular}{r|}
                    $10011011$ \\
                    $10011101$ \\
                    \hline 
                    $100111000$
                \end{tabular}
            }

            \vspace{\Aspace} \par
            c) $1010{.}01 + 10{.}111$
            \\ { \color{azul} 
                \begin{tabular}{r|}
                    $1010{.}010$ \\
                    $10{.}111$ \\
                    \hline 
                    $1101{.}001$
                \end{tabular}
            }



        % - Problema 2
        \vspace{\Pspace}
        \item Represente cada uno de los siguientes números decimales con signo en el sistema de complemento a 2. Use un total de ocho bits, incluyendo el bit de signo. \par
            % Respuestas:
            \vspace{\Aspace} \par
            a) $+1$
            \\ { \color{azul} 00000001}

            \vspace{\Aspace} \par
            b) $-128$
            \\ { \color{azul}
                \begin{tabular}{r|}
                    $10000000$  \\
                    \hline
                    $01111111$  \\
                    $1$         \\
                    \hline 
                    $10000000$
                \end{tabular}
            }

            \vspace{\Aspace} \par
            c) $+169$
            \\ { \color{azul} Dado que este número requiere 8 bits para su representación usaremos un bit extra para el signo. \\ 010101001 }

            \newpage
            d) $0$
            \\ { \color{azul} 00000000 }

            \vspace{\Aspace} \par
            e) $+84$
            \\ { \color{azul} 01010100 }

            \vspace{\Aspace} \par
            f) $-190$
            \\ { \color{azul} 
Dado que este número requiere 8 bits para su representación usaremos un bit extra para el signo. \\
                \begin{tabular}{r|}
                    $010111110$  \\
                    \hline
                    $101000001$  \\
                    $1$         \\
                    \hline 
                    $101000010$
                \end{tabular}
            }
            
            \vspace{\Aspace} \par
            g) $+3$
            \\ { \color{azul} 00000011 }

            \vspace{\Aspace} \par
            h) $-3$
            \\ { \color{azul} 
                \begin{tabular}{r|}
                    $00000011$  \\
                    \hline
                    $11111100$  \\
                    $1$         \\
                    \hline 
                    $11111101$
                \end{tabular}
            }



        % - Problema 3
        \item Cada uno de los siguientes números representa un número decimal con signo en el sistema de complemento a 2. Determine el valor decimal de los siguientes valores:
            % Respuestas:
            \vspace{\Aspace} \par
            a) $10000000$
            \\ { \color{azul} $-2^{7} = -128$}

            \vspace{\Aspace} \par
            b) $11111111$
            \\ { \color{azul} 
                \begin{tabular}{r|}
                    $11111111$  \\
                    \hline
                    $00000000$  \\
                    $1$         \\
                    \hline 
                    $00000001$
                \end{tabular}
                \vspace{\Aspace}\\
                R:/ $11111111 = -1$
            }

            \vspace{\Aspace} \par
            c) $10000001$
            \\ { \color{azul}
                \begin{tabular}{r|}
                    $10000001$  \\
                    \hline
                    $01111110$  \\
                    $1$         \\
                    \hline 
                    $01111111$
                \end{tabular}
                \vspace{\Aspace}\\
                R:/ $10000001 = -127$
            }

            \vspace{\Aspace} \par
            d) $01100011$
            \\ { \color{azul} 99 }

            \vspace{\Aspace} \par
            e) $11011001$
            \\ { \color{azul} 
                \begin{tabular}{r|}
                    $11011001$  \\
                    \hline
                    $00100110$  \\
                    $1$         \\
                    \hline 
                    $00100111$
                \end{tabular}
                \vspace{\Aspace}\\
                R:/ $11011001 = -39$
            }



        % - Problema 4
        \vspace{\Pspace}
        \item La razón por la que el método de signo-magnitud para representar números con signo $n$ se utiliza en la mayoría de las computadoras puede ilustrarse mediante lo siguiente:.
            % Respuestas:
            \vspace{\Aspace} \par
            a) Represente +12 en ocho bits, utilizando la forma signo-magnitud.
            \\ { \color{azul} 00001100 }

            \vspace{\Aspace} \par
            b) Represente -12 en ocho bits, utilizando la forma signo-magnitud.
            \\ { \color{azul} 10001100}

            \vspace{\Aspace} \par
            c) Sume los dos números binarios y observe que la suma no se parece en nada a cero.
            \\ { \color{azul} 
                \begin{tabular}{r|}
                    $00001100$  \\
                    $10001100$  \\
                    \hline 
                    $10011000$
                \end{tabular} \\
En signo-magnitud, el bit más significativo indica el signo (0 positivo, 1 negativo).
Así, +12 y -12 tienen los mismos 7 bits de magnitud (0001100), solo cambia el bit de signo.
Pero al sumarlos, los bits no “se cancelan” como en complemento a 2 — el resultado 10011000 no representa cero, mostrando la gran desventaja del método.
            }


        \vspace{\Pspace}
        % - Problema 5
        \item Multiplique los siguientes pared de números binarios. \par
            % Respuestas:
            \vspace{\Aspace} \par
            a) $111 \times 101$
            \\ { \color{azul} 
                \begin{tabular}{r|}
                    $111$  \\
                    $\times101$  \\
                    \hline 
                    $111$ \\
                    $000\phantom{0}$ \\
                    $+111\phantom{00}$ \\
                    \hline
                    $100011$
                \end{tabular} \\
            }

            \vspace{\Aspace} \par
            b) $1011 \times 1011$
            \\ { \color{azul} 
                \begin{tabular}{r|}
                    $1011$  \\
                    $\times1011$  \\
                    \hline 
                    $1011$ \\
                    $1011\phantom{0}$ \\
                    $0000\phantom{00}$ \\
                    $+1011\phantom{000}$ \\
                    \hline
                    $1111001$
                \end{tabular} \\
            }

            \vspace{\Aspace} \par
            c) $101{.}101 \times 110{.}010$
            \\ { \color{azul} 
                \begin{tabular}{r|}
                    $101{.}101$  \\
                    $\times110{.}010$  \\
                    \hline 
                    ${.}000000$ \\
                    $1{.}01101\phantom{0}$ \\
                    $00{.}0000\phantom{00}$ \\
                    $000{.}000\phantom{000}$ \\
                    $1011{.}01\phantom{0000}$ \\
                    $+10110{.}1\phantom{00000}$ \\
                    \hline
                    $100011{.}001010$
                \end{tabular} \\
            }

            \newpage
            \vspace{\Aspace} \par
            d) $0{.}1101 \times 0{.}1011$
            \\ { \color{azul} 
                \begin{tabular}{r|}
                    $0{.}1101$  \\
                    $\times0{.}1011$  \\
                    \hline 
                    $01101$ \\
                    $01101\phantom{0}$ \\
                    $00000\phantom{00}$ \\
                    ${.}01101\phantom{000}$ \\
                    $+0{.}0000\phantom{0000}$ \\
                    \hline
                    $0{.}10001111$
                \end{tabular} \\
            }


    
        % - Problema 6
        \vspace{\Pspace}
        \item Realice las siguientes divisiones.
            % Respuesta:
            \vspace{\Aspace} \par
            a) $1100 \div 100$
            \\ { \color{azul}  R/: 11}

            \vspace{\Aspace} \par
            b) $111111 \div 1001$
            \\ { \color{azul} R/: 111}

            \vspace{\Aspace} \par
            c) $10111 \div 100$
            \\ { \color{azul} R/: 101.11}

            \vspace{\Aspace} \par
            d) $10110{.}1101 \div 1{.}1$
            \\ { \color{azul} R/: $1111{.}0011\overline{01}$}
            

        
        \newpage
        % - Problema 7
        \item Escriba la tabla de funciones para un medio sumador, $HA$ (entradas $A$ y $B$; salidas SUMA y ACARREO). A partir de la tabla de funciones, diseñe un circuito lógico que actúe como medio sumador.
        \begin{figure}[!ht]
            \centering
            \includegraphics[width=0.5\textwidth]{Circuitos/Figuras/Figura_8.pdf}
        \end{figure}
            % Respuesta:
            \vspace{\Aspace} \par
            { \color{azul} 
                \begin{tabular}{cc|cc}
                    \textbf{A} & \textbf{B} & \textbf{Suma} & \textbf{Acarreo} \\
                    \hline
                    1 & 1 & 0 & 1 \\
                    1 & 0 & 1 & 0 \\
                    0 & 1 & 1 & 0 \\
                    0 & 0 & 0 & 0 \\
                \end{tabular} \\
                \begin{figure}[!ht]
                    \centering
                    \includegraphics[width=0.8\textwidth]{Circuitos/Respuestas/Respuesta_7.pdf}
                \end{figure}
            }



        \newpage
        % - Problema 8
        \vspace{\Pspace}
        \item El desbordamiento ocurre cuando los dos números que se van a sumar o a restar producen un resultado que contiene más bits que la capacidad del acumulador. Diseñe un circuito lógico para el sumador de la figura que produzca una salida de 1 cada vez que ocurra la condición de desbordamiento.
        \begin{figure}[!ht]
            \centering
            \includegraphics[width=0.8\textwidth]{Circuitos/Figuras/Figura_9.pdf}
        \end{figure}
            % Respuesta:
            \vspace{\Aspace} \par
            { \color{azul} }
    \end{enumerate}
\end{document}
