\documentclass[a4paper, 12pt]{article}

% -- Language --
\usepackage[spanish]{babel}
\usepackage[utf8]{inputenc}

% ----- Fonts -----
% -- Fuente --
\usepackage{fontspec}
\setmonofont{JetBrainsMono Nerd Font}  

% -- Color --
\usepackage{xcolor}
%\definecolor{azul}{RGB}{00,33,99}
\definecolor{azul}{RGB}{35,72,180}

% -- Page Margin --
\usepackage[margin=1in]{geometry}

% -- Espaciados --
\newcommand{\Pspace}{0.5cm}
\newcommand{\Aspace}{0.2cm}

% -- Imagenes --
\usepackage{graphicx}
\usepackage{float}

% -- Matemáticas --
\usepackage{amsmath, amssymb}

% -- Gráficas --
\usepackage{pgfplots}
\pgfplotsset{compat=1.18}

% -- Código --
\usepackage{listings}
\lstset{
    language=C++,                   % Lenguaje del código
    basicstyle=\ttfamily\small,     % Fuente del código
    keywordstyle=\color{blue},      % Color de palabras clave
    commentstyle=\color{gray},      % Color de comentarios
    stringstyle=\color{red},        % Color de cadenas
    numbers=left,                   % Números de línea a la izquierda
    numberstyle=\tiny\color{gray},
    breaklines=true,                % Permitir saltos de línea
    frame=single                    % Marco alrededor del código
}


\title
{
    Diseño de Sistemas Digitales 2025-2 \\
    Problemas Unidad 1
}

    \begin{document}

    \maketitle

    \begin{center}
        \begin{tabular}{r|l}
            \textbf{Expediente} & \textbf{Nombre} \\ \hline
            219208106 & Bórquez Guerrero Angel Fernando \\
        \end{tabular}
    \end{center}

    \rule{\linewidth}{0.3mm}



    \vspace{0.3cm}
    \begin{enumerate}
        % - Problema 1
        \item ¿Cuáles de la siguientes son cantidades analógicas y cuáles digitales? \par
            % Respuestas:
            \vspace{\Aspace} \par
            a) El número de átomos en una muestra de material.
            \\ { \color{azul} Digital }

            \vspace{\Aspace} \par
            b) La altitud de una aeronave.
            \\ { \color{azul} Analógica }

            \vspace{\Aspace} \par
            c) La presión en la llanta de una bicicleta
            \\ { \color{azul} Analógica }

            \vspace{\Aspace} \par
            d) La corriente que pasa a través de una bocina.
            \\ { \color{azul} Analógica }

            \vspace{\Aspace} \par
            e) La configuración del temporizador en un horno de microondas.
            \\ { \color{azul} Digital }


        % - Problema 2
        \item ¿Cuáles de las siguientes cantidades son analógicas y cuáles digitales? \par          
            % Respuestas:
            \vspace{\Aspace} \par
            a) La anchura de una pieza de madera
            \\ { \color{azul} Analógica }

            \vspace{\Aspace} \par
            b) La cantidad de tiempo transcurrido antes de que se apague el timbre.
            \\ { \color{azul} Analógica }

            \vspace{\Aspace} \par
            c) La hora del día que se muestra en un reloj de cuarzo.
            \\ { \color{azul} Digital }

            \vspace{\Aspace} \par
            d) La altitud por encima del nivel del mar, si se mide desde una escalera.
            \\ { \color{azul} Digital }

            \vspace{\Aspace} \par
            e) La altura por encima del nivel del mar, si se mide desde una rampa.
            \\ { \color{azul} Analógica }
       

        % - Problema 3
        \item ¿Cuál es el máximo número que podemos contar si utilizamos 14 bits?
            \vspace{\Aspace} \par
            { \color{azul} $2^{14} - 1 = 16384 - 1 = 16383$ }
 

        % - Problema 4
        \item ¿Cuántos bits se necesitan para contar hasta 511
            \vspace{\Aspace} \par
            { \color{azul} $2^{9} -1 = 512 - 1 = 511$, por lo tanto, se necesitan 9 bits. }
        

        % - Problema 5
        \item Convierta los siguientes números binarios en decimales.
            % Respuestas:
            \vspace{\Aspace} \par
            a) 10110
            \\ { \color{azul} 22 }

            \vspace{\Aspace} \par
            b) 10010101
            \\ { \color{azul} 149 }

            \vspace{\Aspace} \par
            c) 100100001001
            \\ { \color{azul} 2313 }

            \vspace{\Aspace} \par
            d) 01101011
            \\ { \color{azul} 107 }

            \vspace{\Aspace} \par
            e) 11111111
            \\ { \color{azul} 255 }

            \vspace{\Aspace} \par
            f) 01101111
            \\ { \color{azul} 111 }


        % - Problema 6
        \item Convierta los siguientes valores decimales en binarios.
            % Respuestas:
            \vspace{\Aspace} \par
            a) 37
            \\ { \color{azul} 100101 }

            \vspace{\Aspace} \par
            b) 13
            \\ { \color{azul} 1101 }

            \vspace{\Aspace} \par
            c) 189
            \\ { \color{azul} 10111101 }

            \vspace{\Aspace} \par
            d) 1000
            \\ { \color{azul} 1111101000 }

            \vspace{\Aspace} \par
            e) 77
            \\ { \color{azul} 1001101 }

            \vspace{\Aspace} \par
            f) 390
            \\ { \color{azul} 110000110 }


        \newpage
        % - Problema 7
        \item Convierta cada número hexadecimal en su equivalente decimal.
            % Respuestas:
            \vspace{\Aspace} \par
            a) 743
            \\ { \color{azul} 1859 }

            \vspace{\Aspace} \par
            b) 36
            \\ { \color{azul} 54 }

            \vspace{\Aspace} \par
            c) 37FD
            \\ { \color{azul} 14333 }

            \vspace{\Aspace} \par
            d) 2000
            \\ { \color{azul} 8192 }

            \vspace{\Aspace} \par
            e) 165
            \\ { \color{azul} 357 }

            \vspace{\Aspace} \par
            f) ABCD
            \\ { \color{azul} 43981 }


        % - Problema 8
        \item Convierta cada uno de los siguientes números decimales en hexadecimales.
            % Respuestas:
            \vspace{\Aspace} \par
            a) 59
            \\ { \color{azul} 3B }

            \vspace{\Aspace} \par
            b) 372
            \\ { \color{azul} 174 }

            \vspace{\Aspace} \par
            c) 919
            \\ { \color{azul} 397 }

            \vspace{\Aspace} \par
            d) 1024
            \\ { \color{azul} 400 }

            \vspace{\Aspace} \par
            e) 771
            \\ { \color{azul} 303 }


        % - Problema 9
        \item Convierta cada uno de los valores hexadecimales del problema 7 en números binarios.
            % Respuestas:
            \vspace{\Aspace} \par
            a) 743
            \\ { \color{azul} 11101000011 }

            \vspace{\Aspace} \par
            b) 36
            \\ { \color{azul} 110110 }

            \vspace{\Aspace} \par
            c) 37FD
            \\ { \color{azul} 11011111111101 }

            \vspace{\Aspace} \par
            d) 2000
            \\ { \color{azul} 10000000000000 }

            \vspace{\Aspace} \par
            e) 165
            \\ { \color{azul} 101100101 }

            \vspace{\Aspace} \par
            f) ABCD
            \\ { \color{azul} 1010101111001101 }


        % - Problema 10
        \item Convierta los números binarios del problema 5 en hexadecimales.
            % Respuestas:
            \vspace{\Aspace} \par
            a) 10110
            \\ { \color{azul} 16 }

            \vspace{\Aspace} \par
            b) 10010101
            \\ { \color{azul} 95 }

            \vspace{\Aspace} \par
            c) 100100001001
            \\ { \color{azul} 909 }

            \vspace{\Aspace} \par
            d) 01101011
            \\ { \color{azul} 6B }

            \vspace{\Aspace} \par
            e) 11111111
            \\ { \color{azul} FF }

            \vspace{\Aspace} \par
            f) 01101111
            \\ { \color{azul} 6F }


        % - Problema 11
        \item Codifique los siguientes números decimales en BCD.
            % Respuestas:
            \vspace{\Aspace} \par
            a) 47
            \\ { \color{azul} 0100 0111 }

            \vspace{\Aspace} \par
            b) 962
            \\ { \color{azul} 1001 0110 0010 }

            \vspace{\Aspace} \par
            c) 187
            \\ { \color{azul} 0001 1000 0111 }

            \vspace{\Aspace} \par
            d) 6727
            \\ { \color{azul} 0110 0111 0010 0111 }

            \vspace{\Aspace} \par
            e) 13
            \\ { \color{azul} 0001 0011 }

            \vspace{\Aspace} \par
            f) 529
            \\ { \color{azul} 0101 0010 1001 }


        % - Problema 12
        \item ¿Cuántos bits se requieren para representar los números decimales en el intervalo de 0 a 999 si se utiliza:
            % Respuestas:
            \vspace{\Aspace} \par
            a) Codigo binario directo
            \\ { \color{azul} 10 }

            \vspace{\Aspace} \par
            b) Código BCD
            \\ { \color{azul} 12 }

 
        % - Problema 13
        \item Los siguientes números están en BCD. Conviéralos en decimales.
            % Respuestas:
            \vspace{\Aspace} \par
            a) 1001 0111 0101 0010
            \\ { \color{azul} 9752 }

            \vspace{\Aspace} \par
            b) 0001 1000 0100
            \\ { \color{azul} 184 }

            \vspace{\Aspace} \par
            c) 0110 1001 0101
            \\ { \color{azul} 695 }

            \vspace{\Aspace} \par
            d) 0111 0111 0111 0101
            \\ { \color{azul} 7775 }

            \vspace{\Aspace} \par
            e) 0100 1001 0010
            \\ { \color{azul} 492 }

            \vspace{\Aspace} \par
            f) 0101 0101 0101
            \\ { \color{azul} 555 }


        % - Problema 14
        \item Responda las preguntas:
            % Respuestas:
            \vspace{\Aspace} \par
            a) ¿Cuántos bits hay en ocho bytes?
            \\ { \color{azul} 64 }

            \vspace{\Aspace} \par
            b) ¿Cuál es el número hexadecimal más grande que puede representarse en cuatro bytes?
            \\ { \color{azul} $4$ Bytes $= 32$ bits $= 100000000_{16}$}

            \vspace{\Aspace} \par
            c) ¿Cuál es el valor decimal codificado en BCD más grande que puede representarse en tres bytes?
            \\ { \color{azul} 999999 }


        % - Problema 15
        \item Responda las preguntas:
            % Respuestas:
            \vspace{\Aspace} \par
            a) ¿Cuántos nibbles pueden almacenarse en una palabra de 16 bits?
            \\ { \color{azul} 4 nibbles }

            \vspace{\Aspace} \par
            b) ¿Cuántos bytes se requieren para formar una palabra de 24 bits?
            \\ { \color{azul} 3 bytes }
    \end{enumerate}
\end{document}
