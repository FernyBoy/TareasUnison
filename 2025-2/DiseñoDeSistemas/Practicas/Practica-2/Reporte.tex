
\documentclass[a4paper,12pt]{article}
\usepackage[spanish]{babel}
\usepackage[utf8]{inputenc}
\usepackage{graphicx}
\usepackage{amsmath}
\usepackage{geometry}
\usepackage{caption}
\geometry{margin=2.5cm}

\begin{document}

\begin{center}
    {\LARGE \textbf{Sumador Completo de 5 Bits en Logisim-Evolution}}\\[0.5cm]
    \textbf{Diseño de Sistemas Digitales}\\
    \textbf{Bórquez Guerrero Angel Fernando - 219208106}\\
    \textbf{Fecha: \today}
\end{center}

\vspace{1cm}

\section*{Descripción general}
El objetivo de esta práctica es diseñar un \textbf{sumador completo de 5 bits} utilizando el software \textit{Logisim-Evolution}.  
Para realizar la suma de dos números binarios de 5 bits, se emplean cinco módulos de \textit{sumadores completos (Full Adders, FA)} conectados en paralelo.  
Cada FA se encarga de sumar los bits correspondientes de las entradas \( A_i \) y \( B_i \), junto con el acarreo de entrada (\( C_{in} \)) proveniente del bit anterior.

\vspace{0.5cm}

\section*{Sumador completo (FA)}
El sumador completo es el bloque básico del circuito.  
Cuenta con tres entradas: \( A \), \( B \) y \( C_{in} \); y dos salidas: la \textbf{suma} (\( S \)) y el \textbf{acarreo de salida} (\( C_{out} \)).  
Su comportamiento se define mediante las siguientes expresiones lógicas:

\[
S = A \oplus B \oplus C_{in}
\]
\[
C_{out} = (A \cdot B) + (C_{in} \cdot (A \oplus B))
\]

A continuación, se muestra el circuito implementado en Logisim-Evolution:

\begin{center}
    \includegraphics[width=0.8\textwidth]{FA.png}
    \captionof{figure}{Circuito del sumador completo (FA).}
\end{center}

\vspace{0.5cm}

\section*{Sumador completo de 5 bits}
Para realizar la suma de dos números binarios de 5 bits, se conectaron cinco módulos FA.  
El \textbf{acarreo de salida} (\( C_{out} \)) de cada módulo se conecta al \textbf{acarreo de entrada} (\( C_{in} \)) del siguiente, comenzando con un acarreo inicial igual a cero.

El resultado final del circuito incluye:
\begin{itemize}
    \item Cinco salidas de suma (\( S_0, S_1, S_2, S_3, S_4 \))
    \item Un acarreo de salida global (\( C_{out} \))
\end{itemize}

\begin{center}
    \includegraphics[width=0.9\textwidth]{Circuito.png}
    \captionof{figure}{Sumador completo de 5 bits usando cinco módulos FA.}
\end{center}

\vspace{0.5cm}

\section*{Conclusión}
El diseño del sumador completo de 5 bits demuestra cómo un sistema más complejo puede construirse a partir de bloques básicos como el FA.  
Este enfoque modular facilita la comprensión y el diseño de circuitos digitales de mayor escala.  
El circuito cumple correctamente con la operación de suma binaria, propagando el acarreo de manera adecuada entre los distintos bits.

\end{document}
