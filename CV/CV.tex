\documentclass[11pt]{article}

%====================
% Basic setup
%====================
\usepackage[left=0.7in, right=0.7in, top=0.7in, bottom=0.7in]{geometry}
\usepackage{hyperref}
\usepackage{xcolor}
\usepackage{enumitem}
\usepackage{titlesec}
\usepackage{setspace}

\hypersetup{
    colorlinks=true,
    urlcolor=black
}

\setlength{\parindent}{0pt}
\setlength{\parskip}{4pt}
\pagenumbering{gobble}

%====================
% Section formatting
%====================
\titleformat{\section}
  {\large\bfseries}
  {}
  {0pt}
  {}
  [\vspace{2pt}\titlerule]

%====================
% Document
%====================
\begin{document}

%====================
% Header
%====================
{\Huge \textbf{Angel Fernando Borquez Guerrero}} \\
\vspace{-0.4cm}
\par \href{https://www.linkedin.com/in/fernandoborquez/}{LinkedIn} \quad | \quad
\href{https://github.com/FernyBoy}{GitHub} \quad | \quad
\href{mailto:fernandoborquez215@icloud.com}{fernandoborquez215@icloud.com} \quad | \quad
+52 633 100 5991 \\
Hermosillo, Sonora.

%====================
% Education
%====================
\section*{Education}
\textbf{Universidad de Sonora} \hfill GPA: 90\%\\
Bachelor of Science in Computer Science \hfill \textbf{August 2023 -- Present}

%====================
% Professional Experience
%====================
\section*{Professional Experience}
\textbf{ACARUS — Supercomputing Area} \hfill Supervisor: Dra. María del Carmen Heras Sánchez\\
Professional Intern \hfill \textbf{September 2025 -- Present}
\vspace{-0.3cm}
\begin{itemize}[leftmargin=*]
    \item Professional internship in the Supercomputing area at the University of Sonora.
    \vspace{-0.3cm}
    \item Part of a development team building a specialized forum for users of \textit{Yuca}, the university HPC system.
    \vspace{-0.3cm}
    \item Implementing features such as user posts, direct messaging, and service request workflows.
    \vspace{-0.3cm}
    \item Supporting an active HPC user community by improving communication and technical support processes.
\end{itemize}

%====================
% Research Experience
%====================
\section*{Research Experience}
\textbf{University of Guadalajara} \hfill Supervisor: Dr. Rafael Morales Gamboa \\
Research Intern \hfill \textbf{June -- July 2025}
\vspace{-0.3cm}
\begin{itemize}[leftmargin=*]
    \item Extended a validated Entropic Associative Memory (EAM) model through large-scale experimental validation and robustness analysis.
    \vspace{-0.3cm}
    \item Adapted and refactored the data pipeline to support class balancing, dynamic class selection, and scalable dataset handling.
    \vspace{-0.3cm}
    \item Retrained perceptual neural networks (autoencoder and classifier) to generate feature representations compatible with the EAM architecture.
    \vspace{-0.3cm}
    \item Designed and implemented a novelty detection experiment to evaluate the model’s ability to reject unseen classes.
    \vspace{-0.3cm}
    \item Introduced experimental controls enabling partial memory registration and quantitative analysis of rejection behavior via no-response metrics.
    \vspace{-0.3cm}
    \item Refactored the system into a flexible experimental framework allowing controlled evaluations with a variable number of classes.
\end{itemize}


%====================
% Competitions & Projects
%====================
\section*{Competitions \& Projects}
\textbf{NASA Space Apps Challenge (2024)} \hfill Project: Voyager CXXV\\
Project Co-Lead \hfill \textbf{October 2024}
\vspace{-0.3cm}
\begin{itemize}[leftmargin=*]
    \item Co-led the development of a geospatial simulation tool to evaluate the impact of urban green areas on temperature and air quality in Hermosillo, Mexico.
    \vspace{-0.3cm}
    \item Designed and implemented a system allowing users to define custom zones and simulate long-term environmental effects of urban reforestation.
    \vspace{-0.3cm}
    \item Integrated spatial data visualization through heat maps to support data-driven insights for urban planning scenarios.
    \vspace{-0.3cm}
    \item Coordinated technical tasks within the team, contributing to project planning, feature definition, and final integration.
\end{itemize}

\textbf{TE AI CUP (2025)} \hfill Project: AI Generated Machine Downtime Insights\\
UI Developer \hfill \textbf{January -- May 2025}
\vspace{-0.3cm}
\begin{itemize}[leftmargin=*]
    \item Developed the user interface for an AI-based industrial downtime analysis tool, focusing on clear visualization of patterns, anomalies, and clustering results derived from machine time-series data.
    \vspace{-0.3cm}
    \item Designed an accessible and intuitive frontend to enable non-technical users to explore insights generated by unsupervised learning models.
\end{itemize}

\textbf{NASA Space Apps Challenge (2025)} \hfill Project: Endurance CXXV\\
Project Co-Lead \hfill \textbf{October 2025}
\vspace{-0.3cm}
\begin{itemize}[leftmargin=*]
    \item Co-led the development of an educational weather application designed for a children-oriented audience, focused on climate awareness and engagement.
    \vspace{-0.3cm}
    \item Developed both frontend and backend components of the application, contributing to core functionality and system integration.
    \vspace{-0.3cm}
    \item Designed and implemented the user interface with an emphasis on accessibility, usability, and age-appropriate interaction.
\end{itemize}

\textbf{1st Place Winner -- MICAI 2025 MeDA Challenge} \hfill \textbf{October 2025}
\vspace{-0.3cm}
\begin{itemize}[leftmargin=*]
    \item Contributed to a 1st Place victory in a national academic competition focused on Medical Domain Adaptation using Self-Supervised Learning (SSL), leading to an invitation to publish in a special issue of the \textit{Health Information Science and Systems} journal.
    \vspace{-0.3cm}
    \item Structured and organized the dataset used for model training by designing a clear directory hierarchy to support reproducible and efficient experimentation.
    \vspace{-0.3cm}
    \item Collaborated in the writing and revision of the scientific article, contributing to the methodological description and overall presentation of results.
    \vspace{-0.3cm}
    \item Developed visual and presentation assets used to communicate the project during the competition and dissemination stages.
\end{itemize}


%====================
% Technical Skills
%====================

\section*{Technical Skills}
\textbf{Programming:} Python, C++, Java, SQL, JavaScript, TypeScript\\
\textbf{Web Development:} HTML, CSS, SASS, Vue.js, Django, NPM\\
\textbf{Systems \& Virtualization:} Linux (Arch Linux), High Performance Computing (HPC), Virtual Machines, Proxmox VE, VirtualBox\\
\textbf{Data \& Research Tools:} Jupyter, Google Colab, LaTeX, GIS tools\\
\textbf{Version Control \& Platforms:} Git, GitHub


\end{document}
